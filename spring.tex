\documentclass[12pt]{article}
\usepackage[utf8]{inputenc}
\usepackage[russian]{babel}
\usepackage{amssymb}
\usepackage{systeme}
\usepackage{amsmath}
\usepackage{amsthm}
\usepackage{graphicx}
\usepackage{mathtools}
\usepackage{bbold}
\usepackage{enumitem}
\usepackage{collectbox}
\usepackage{multicol}
\usepackage[margin=0.5in]{geometry}
\usepackage{tabularx}
\usepackage[scr=boondoxo,scrscaled=1.05]{mathalfa}
\usepackage{stackengine}
\usepackage{wrapfig}
\usepackage{relsize}
\usepackage{latexsym}

\title{Конспекты по линейной алгебре, 2 сем}
\author{Пак Александр}

%----ENVIRONMENTS--------------%
\newtheorem{theorem}{Теорема}[subsection]
\renewcommand{\thetheorem}{\arabic{theorem}}
\newtheorem{lemma}{Лемма}[subsection]
\renewcommand{\thelemma}{\arabic{lemma}}
\newtheorem{defin}{Определение}[subsection]
\renewcommand{\thedefin}{\arabic{defin}}
\newtheorem*{stat}{Утверждение}
\newtheorem{corollary}{Следствие}[theorem]
\renewcommand{\thecorollary}{\arabic{corollary}}

\newenvironment{mylist}{\begin{enumerate}[noitemsep, nolistsep]}{\end{enumerate}}

\theoremstyle{remark}
\newtheorem*{remark}{Замечание}

\theoremstyle{definition}
\newtheorem*{examples}{Примеры}

%-------------------------------%

%------COMMANDS-----------------%
\newcommand{\N}{\numberset{N}}
\newcommand{\Q}{\mathbb Q}
\newcommand{\R}{\mathbb R}
\newcommand{\Z}{\mathbb Z}
\newcommand{\0}{\mathbb{0}}
\newcommand{\mybox}{
	\collectbox{
		\setlength{\fboxsep}{1pt}
		\fbox{\BOXCONTENT}
	}
}
\newcommand{\E}{\mathcal{E}}
\newcommand{\A}{\mathcal{A}}
\newcommand{\B}{\mathcal{B}}
\let\vec\overline
\newcommand{\p}{\mathcal{P}}
\newcommand{\pu}{\sqsupset}
%-------------------------------%

\graphicspath{ {./imgs/} }
\setlength{\parindent}{0ex}
\linespread{1.2}

\begin{document}
	\pagenumbering{gobble}
	\maketitle
	\newpage
	\pagenumbering{arabic}
	\setcounter{section}{6}
	\tableofcontents
	\newpage
	
	\section{Линейные отображения}
	
	\subsection{Основные определения}
	\begin{defin}
		$U, V$ -- линейные пространства над полем $K(\R/ \mathbb{C})$\\
		Линейным отображением $\A$ называется $\A: U \rightarrow V$, обладающее свойством линейности:\\
		$\forall \lambda \in K, \forall u, v \in U\\ 
		\A(u + \lambda v) = \A(u) + \lambda \A(v)$
	\end{defin}
	
	\begin{remark}\hfill
		\begin{mylist}
			\item Записываем не $\A(u)$, а $\A u$
			\item "Поточечно" выполняются все арифметические операции, свойственные функциям
			\item $\A\0_U = \0_V$
		\end{mylist}
	\end{remark}
	\begin{examples}\hfill
		\begin{mylist}
			\item $\0$ -- нулевое отображение $U \rightarrow V$\\
			$\forall u \in U: \0u = \0_v$
			\item $\mathcal{E}$ -- тождественное отображение: $V \rightarrow V$\\
			$\forall v \in V: \mathcal{E}v = v$ 
			\item $U = V = P_n$ -- многочлены степени до $n$
			\\$\A: V \rightarrow V$\\
			$\A p = p'(t)$ -- дифференциальный оператор\\
			$\A(p_1 + \lambda p_2) = (p_1 + \lambda p_2)' = p_1' + \lambda p_2' = \A p_1 + \lambda \A p_2$\\
			Линейное отображение 
			$\A = \frac{d}{dt}$
			\item $U = \R^n \ V = \R^m\\
			\A = (a_{i j})_{m\times n}\\
			\A: x\in U \rightarrow y = \A x \in V\\
			x_1 + \lambda x_2 \in \R^n \rightarrow y = \A(x_1 + \lambda x_2) = \A x_1 + \lambda \A x_2$
			\item $U \cong V$. То есть отображение, на котором строится изоморфизм является линейным.
		\end{mylist}
	\end{examples}
	\begin{defin}$\lambda \in K \ \A: U\rightarrow V$\\
		Произведение линейного отображения на скаляр называется линейное отображение \\
		$\B = \lambda \A \\
		\B: U \rightarrow V \ \forall u \in U \ \B u = \lambda \A u$
	\end{defin}
	\begin{defin}
		Суммой линейных отображений $\A, \B: U \rightarrow V$ называется $\mathcal{C}: U \rightarrow V\\
		\forall u \in U \ \mathcal{C}u = \A u + \B u$
		\mybox{
			$\mathcal{C} = \A + \B$
		}
	\end{defin}
	
	\begin{defin}
		$-\A$ -- отображение противоположное $\A$\\
		$\forall u \in U \ (-\A)u = -1 \cdot \A u$
	\end{defin}
	
	$L(U, V) = Hom_K(U, V) = Hom(U, V) = \mathcal{L}(U, V)$\\
	$L(U, V)$ -- множество всех линейных отображений из $U$ в $V$.\\
	Линейное отображение = гомоморфизм с операциями $\lambda \A$ и $\A + \B$\\
	Выполнены свойства 1--8 линейного пространства (проверить самим). \\Значит \mybox{$L(U, V)$ -- линейное пространство}
	\begin{defin}
		$\A \in L(U, V)\\
		Ker\A = \{u \in U \ | \ \A u = \0_v\}$ -- ядро линейного отображения. 
	\end{defin}
	\begin{defin}
		$Im\A =\{v \in V = \A u \ \forall u \in U\} = \\ \{v \in V \ | \  \exists u \in U \ v = \A u\}$ --
		образ линейного отображения.
	\end{defin}
	\includegraphics[width=250px]{1}
	
	Упр: $Ker\A$ и $Im\A$ - это подпространства соответственно пространств $U$ и $V$. То есть они замкнуты относительно линейных операций. \\
	Если $Ker\A$ конечномерное подпространство $U$, то \\
	\mybox{$dim \ Ker\A = def\A$} -- дефект линейного отображения.\\
	Если $Im\A$ конечномерное подпространство $V$, то \\
	\mybox{$dimIm\A = rg\A$} -- ранг линейного отображения.
	
	\begin{stat}
		$\A$ изоморфно между $U$ и $V \Leftrightarrow$ 
		\begin{mylist}
			\item $\A \in L(U, V)$
			\item $Im\A = V$
			\item $Ker\A = \{\0\}$ тривиально 
		\end{mylist}
	\end{stat}
	\begin{proof}
		$\A$ изоморфно $\Leftrightarrow$ взаимнооднозначное соответствие + линейность -- $\A\in L(U, V)$\\
		$\0_u \leftrightarrow \0_v$, т. к. изоморфизм $\Rightarrow Ker\A = \{\0\}$\\
		Пусть $Ker\A = \{\0\}$\\
		Докажем инъективность $v_1 = v_2 \Leftrightarrow u_1 = u_2\\
		v_1 = \A u_1 \ v_2 = \A u_2\\
		\0 = v_1-v_2 = \A u_1 - \A u_2 = \A(u_1 - u_2) = \0$ т. к. ядро тривиально.\\
		Сюръективность. $Im\A = V \Leftrightarrow \forall v \in V: \exists u \in U \A u = v$. Последнее и означает сюръекцию.
	\end{proof}
	\begin{defin}
		$\A \in L(U, V)$\\
		--инъективно, если $Ker\A = \{\0 \}$\\
		--сюръективно, если $Im \A = v$\\
		--биективно $\equiv$ изоморфизм, если инъекция + сюръекция.\\
		--эндоморфизм $\equiv$ линейный оператор, если $U \equiv V$\\
		$End_k(V) = End(V) = L(V, V)$\\
		--автоморфизм $\equiv$ эндоморфизм + изоморфизм. \\
		$Aut_k(V) = Aut(V)$
	\end{defin}
	\begin{defin}
		Произведением линейных отображений $\A, \B$ \\
		$\A \in L(W, V) \ \  \B \in L(U, W) \ \
		U \xrightarrow{\B} W \xrightarrow{\A} V$\\
		называется $\mathcal{C} \in L(U, V) : \mathcal{C} = \A\cdot \B$, которое является композицией функций, определяющих отображения $\A$ и $\B$.\\
		$\A\cdot \B = \A \circ \B\\
		\forall u \in U: (\A \B)u = \A(\B u)$\\
		Очевидно, $\mathcal{C}$ -- линейное отображение.
	\end{defin}
	$\Omega \xrightarrow{\mathcal{C}} U \xrightarrow{\B_{1, 2}} W \xrightarrow{\A_{1, 2}} V$\\
	Упр: \begin{mylist}
		\item $\A, \B$ изоморфизмы $\Rightarrow \A\cdot \B$ изоморфизм
		\item $(\A_1 + \A_2)\B = \A_1\B + \A_2\B\\
		\A(\B_1 + \B_2) = \A \B_1 + \A \B_2$ -- дистрибутивность
		\item $\A(\B\mathcal{C}) = (\A \B)\mathcal{C}$ -- ассоциативность
		\item $\lambda \A \B = \A \lambda \B$
	\end{mylist}
	$End(V)$ -- ассоциативная унитарная алгебра\\
	$\mathcal{E}$ -- единица $\mathcal{E}\A = \A\mathcal{E}$
	\begin{defin}
		$\A \in L(U, V)$ изоморфно.\\
		$\forall v \in V \exists! u \in U: v = \A u\\
		\A^{-1}: V \rightarrow U$\\
		\mybox{$\A^{-1}v = u$}\\
		Упр: $\A^{-1}\in L(V, U)$\\
		$\A^{-1} \A = \E_v \ \ \A\A^{-1} = \E_u$
		
	\end{defin}
	$\A\in End(U)$ -- линейный оператор
	
	$\A^{-1}\in End(V)$ -- обратный оператор
	\begin{defin}
		$U_0 \subset U \ \ \A\in L(U, V)$\\
		Сужением линейного отображения $\A$ на линейное подпространство $U_0$ называется \\
		$\A|_{U_0}: U_0 \rightarrow V \ \ \forall u \in U_0 \ \A|_{U_0} u = \A u$
	\end{defin}
	\begin{stat}
		$\A$ изоморфизм $\in L(U, V) \Rightarrow  \A|_{U_0}\in L(U_0, Im(\A|_{U_0}))$ -- изоморфизм
	\end{stat}
	\begin{examples}
		\hfil
		\begin{mylist}
			\item $\0: U\rightarrow U$ -- не сюръекция, не инъекция, эндоморфизм, не автоморфизм.
			\item $\E: U \rightarrow U$ -- автоморфизм
			\item 
			$\A = \frac{d}{dt}$ \  
			$\A: P_n \rightarrow P_n$ -- эндоморфизм, не инъекция, не сюръекция.
			\item $x\in \R^n \rightarrow y = \A x \in \R^n$ -- эндоморфизм. \\
			Сюръекция $\Leftrightarrow rg\A = n \Leftrightarrow \exists \A^{-1} \Leftrightarrow$ инъекция.\\
			То есть автоморфизм.
		\end{mylist}
	\end{examples}
	\begin{theorem}[о $rg$ и $def$ линейного отображения]
		$\A\in L(U, V)$\\
		\mybox{
			$rg\A + def\A = dimU$	
		}
	\end{theorem}
	\includegraphics[width=250px]{2}
	\begin{proof}
		$U_0 = Ker\A$\\
		Дополним линейное пространство $U_1$ до пр-ва $U$:\\
		$U = U_0 \oplus U_1 \ \ \ U_1 \cap U_0 = \{ \0 \}\\
		\forall u \in U : u = u_0 + u_1$ (единственным образом)\\
		$\A u = \A u_0 + \A u_1 = \A u_1 \ \ \ \ Im\A = \A(U_1)\\
		\A_1 = \A|_{U_1}: U_1 \rightarrow Im\A$ \\
		$\A_1$ -- изоморфизм? $Im\A_1 = Im\A$ -- сюръекция\\
		$\left.\begin{array}{c}
		\forall w \in Ker\A_1 \in U_1\\
		KerA_1 \subset KerA = U_0
		\end{array}\right \}
		\Rightarrow w \in U_1 \cap U_0 = \{\0\} \Rightarrow Ker\A_1 = \{\0\}\Rightarrow \A_1$ изоморфизм.\\
		$U_1 \cong Im\A \Leftrightarrow dimU_1 = dim(Im\A)$ -- инъекция.\\
		Т. к. $U = U_0 \oplus U_1$, то $dimU = dimU_0 + dimU_1 = \underset{def\A}{dimKer\A} + \underset{rg\A}{dimIm\A}$
	\end{proof}
	\begin{corollary}[Характеристика изоморфизма]
		\ \\
		$\A\in L(U, V)$ Следующие условия эквивалентны:
		\begin{mylist}
			\item $\A$ изоморфно
			\item $dimU = dimV = rg\A$
			\item $dimU = dimV\\
			Ker\A = \{\0\} \Leftrightarrow def\A = 0$
		\end{mylist}
	\end{corollary}
	\begin{corollary}
		$\A \in End(V)$ Следующие условия эквивалентны:
		\begin{mylist}
			\item $\A \in Aut(V)$
			\item $dimV =rg \A$
			\item $Ker\A = \{\0\} \Leftrightarrow def\A = 0$ 
		\end{mylist}
	\end{corollary}
	
	\subsection{Матрица линейного отображения. Изоморфизм алгебр. Преобразование матрицы линейного отображения при замене базиса.}
		$\A \in L(U, V)\\
		\xi_1 \ldots \xi_n$ базис $U$\\
		$\eta_1 \ldots \eta_m$ базис $V$\\
		$\forall u \in U \ u = \sum\limits_{i=1}^{n} u_i\xi_i \leftrightarrow u = \left(
		\begin{array}{c}
		u_1 \\
		\vdots\\
		u_n 
		\end{array} \right)
		\in \R^n(\mathbb{C}^n)$\\
	\begin{tabular}{cc}
		$\A u = \A(\sum\limits_{i = 1}^{n}u_i\xi_i) = \sum\limits_{i=1}^{n}u_i\A\xi_i$ &
		Достаточно знать, как $\A$ работает на базисных векторах $\xi_1\ldots\xi_n$
	\end{tabular}\\ \\
	$Im\A = span(\A\xi_1, \A\xi_2, \ldots, \A\xi_n)$\\
	$$\A\xi_i \in V = \sum_{j=1}^{m}a_{ji}\eta_j \leftrightarrow A_i = \left(
		\begin{array}{c}
			a_{1i}\\
			a_{2i}\\
			\vdots\\
			a_{mi}
		\end{array}\right)
	\in \R^m(\mathbb{C^m})\ \ \ \ 
	a_{ji} \in \R(\mathbb{C})
	$$\\
	\mybox{
		$A = (A_1 \ldots A_i \ldots A_n) = (a_{ij})_{m\times n}$
	}
	матрица линейного отображения $\A$ относительно базисов $(\xi, \eta)$\\\\
	Частный случай: $\A\in End(V): \underset{e_1\ldots e_n}{V} \rightarrow \underset{e_1\ldots e_n}{V}\\
	A = (a_{ji})_{n\times n}$ -- матрица линейного оператора\\
	$A e_i = \sum\limits_{j=1}^{n}a_{ji}e_j$
	\begin{examples}
		\ \\
		\begin{mylist}
			\item
			$\E:\underset{e_1 \ldots e_n}{V} \rightarrow \underset{e_1 \ldots e_n}{V} \ \ \ 
			\E e_i = e_i \leftrightarrow
			\left( 
			\begin{array}{c}
				0\\
				0\\
				\vdots\\
				1\\
				\vdots\\
				0
			\end{array}
			\right)
			\leftrightarrow E_{m\times n} = \left(
				\begin{array}{ccc}
					1 & \ldots & 0\\
					\ldots & 1 & \ldots\\
					0 & \ldots & 1
				\end{array}
			\right)
			$
			\item 
			$$\E: \underset{e'_1\ldots e'_n}{V} \rightarrow \underset{e_1\ldots e_n}{V}$$
			$$
			\E e'_i =  \sum_{j = 1}^{n}t_{ji}e_j \leftrightarrow T_i = 
			\left(
			\begin{array}{c}
			t_{1i}\\
			\vdots\\
			t_{ni}
			\end{array}
			\right)$$
			$$
			[ \E ] _e = T = 
			\left(
			\begin{array}{c}
			t_{1i}\\
			\vdots\\
			t_{ni}
			\end{array}
			\right)
			= T_{e\rightarrow e'}
			$$
			\item \ \\
			\begin{multicols}{2}
				\includegraphics[width=150px]{3}\\
				$\A: \R^2 \rightarrow \R^2\\
				v = \A u$\\
				Поворот векторов в плоскости на угол $\alpha$. \\
				Очевидно, линейный оператор.
			\end{multicols}
			\begin{multicols}{2}
				\includegraphics[width=150px]{4}\\
				$
				\A_i = \cos\alpha i + \sin\alpha j \leftrightarrow 
				\left(
				\begin{array}{c}
				\cos\alpha\\
				\sin\alpha
				\end{array}
				\right)\\
				\A_j = -\sin\alpha i \cos\alpha j \leftrightarrow
				\left(
				\begin{array}{c}
				-\sin\alpha\\
				\cos\alpha
				\end{array}
				\right)\\
				\A \leftrightarrow A = 
				\left(
				\begin{array}{cc}
				\cos\alpha & -\sin\alpha\\
				\sin\alpha & \cos\alpha
				\end{array}
				\right)
				$
			\end{multicols}
			\item 
			\begin{multicols}{2}
			$\A: \overset{1, t, t^2}{p_2} \rightarrow  \overset{1, t, t^2}{p_2}\\
			\A = \frac{d}{dt}\\
			\A1 = 1' = 0 \leftrightarrow 
			\begin{pmatrix}
				0\\0\\0
			\end{pmatrix}
			\\
			\A t = t' = 1 \leftrightarrow
			\begin{pmatrix}
				1\\0\\0
			\end{pmatrix}
			\\
			\A t^2 = (t^2)' = 2t \leftrightarrow 
			\begin{pmatrix}
				0\\2\\0
			\end{pmatrix}\\
			\A \underset{(1, t, t^2)}{\leftrightarrow}
			\begin{pmatrix}
				0 & 1 & 0\\
				0 & 0 & 2\\
				0 & 0 & 0
			\end{pmatrix}
			$
			\end{multicols}
			$\A: \underset{1, t, t^2}{p_2} \rightarrow \underset{1, t}{p_1}\\
			\A = \frac{d}{dt} \leftrightarrow A = \begin{pmatrix}
				0 & 1 & 0\\
				0 & 0 & 2
			\end{pmatrix}$
	\end{mylist}
	\end{examples}
	\begin{stat}
		$L(U, V) \cong M_{m\times n}$\\
		(Линейное пространство матриц с вещ.(компл.) элементами размерности $m\times n$. 
	\end{stat}	
	\begin{proof}
		Изоморфизм $\equiv$ биекция + линейность.\\
		Биекция. $\A \rightarrow A_{m\times n}$ -- поняли, как сопоставлять.\\
		Теперь обратно. Пусть $A_{m\times n} = (a_{ij})$\\
		\begin{tabular*}{\textwidth}{c @{\extracolsep{100pt}} c}
			$U \ \xi_1\ldots \xi_n$ базис & $\A: U\rightarrow V$\\
			$V \ \eta_1\ldots \eta_m$ базис &
			$\A\xi_i = \sum\limits_{j=1}^{m}a_{ji}\eta_j \in V$
		\end{tabular*}\\
		$
		\forall u \in U \ u = \sum\limits_{i = 1}^{n}u_i\xi_i
		\\
		\A u = \sum\limits_{i = 1}^{n}u_i\A\xi_i \in V \Rightarrow \A \in L(U, V)
		$
		$\A, \B \leftrightarrow A, B\\
		\forall \lambda \in K \ \A + \lambda\B \overset{?}{\leftrightarrow} A + \lambda B$\\
		$$ (\A + \lambda\B)\xi_i = \A\xi_i + \lambda\B\xi_i = \sum_{j = 1}^{m}a_{ji}n_j + \lambda
		\sum_{j = 1}^{m}b_{ji}\eta_j = \sum_{j=1}^{m}(a_{ji} + \lambda b_{ji})\eta_j \leftrightarrow
		c_i = A_i + \lambda B_i \leftrightarrow A + \lambda B \Rightarrow$$ линейность $\Rightarrow$ изоморфизм.
	\end{proof}
	$\A + \lambda \B \leftrightarrow A + \lambda B\\
	\A\B \leftrightarrow A\cdot B\\
	\A, \A^{-1} \leftrightarrow A, A^{-1}\\
	End(V) \cong M_{n\times n}$ -- ассоциативные унитарные алгебры. (Координатный изоморфизм).\\
	Алгебры изоморфны, т.к. сохраняются свойства дистрибутивности, ассоциативности и т. д.
	\\
	
	\textit{Я не особо понял, что мы дальше делаем, но у меня это записано}\\
	\begin{tabular}{cc}
		$U \xi_1\ldots\xi_n$ & $\forall u \in U \leftrightarrow u = \begin{pmatrix}
		u_1\\
		\vdots\\
		u_n
		\end{pmatrix}$\\
		$V \eta_1\ldots\eta_m$ &
		$u = \sum\limits_{i=1}^{n}u_i\xi_i$
	\end{tabular}
	
	$\forall v\in V \leftrightarrow v = \begin{pmatrix}
		v_1\\
		\vdots\\
		v_m
	\end{pmatrix}\\	
	v = \sum\limits_{j=1}^{m}v_j\eta_j\\
	\A\in L(U, V) \underset{\xi, \eta}{\leftrightarrow} A\\
	\sum\limits_{j=1}^m v_j\eta_j = v = \A u = \sum\limits_{i=1}^n u_i\A\xi_i = \sum\limits_{i=1}^n u_i
	\sum\limits_{j=1}^m a_{ji}\eta_j = \sum\limits_{j=1}^m (\sum\limits_{i=1}^n u_i a_{ji})\eta_j$\\
	Так как координаты определяются единственным образом:\\
	\mybox{ 
		$ v_j = \sum\limits_{i=1}^n a_{ji}u_i $ 
	} 
	$\leftrightarrow$ 
	\mybox{$v = Au$} $\leftrightarrow v = \A u$

	\begin{examples}\ \\
		\begin{mylist}
			\item 
			$\A$ поворот на угол $\alpha$\\
			$(i, j) \leftrightarrow A = \begin{pmatrix}
				\cos\alpha & -\sin\alpha\\\sin\alpha & \cos\alpha
			\end{pmatrix}$
			\\
			\begin{tabular}{lcc}
				\includegraphics[width=200px]{5} &
				$\alpha=45^{\circ}$ &
				$A = \begin{pmatrix}
				\frac{\sqrt2}{2} & -\frac{\sqrt2}{2}\\
				\frac{\sqrt2}{2} & \frac{\sqrt2}{2}\end{pmatrix}$\\
			$\mathscr{u} \leftrightarrow u = \begin{pmatrix}2\\1\end{pmatrix}$
			\end{tabular}
			\\
			$\mathscr{v} = \A \mathscr{u} \leftrightarrow v = Au = \begin{pmatrix}
				\frac{\sqrt2}{2} & -\frac{\sqrt2}{2}\\
				\frac{\sqrt2}{2} & \frac{\sqrt2}{2}
			\end{pmatrix}\begin{pmatrix}2\\1\end{pmatrix} = \begin{pmatrix}
				\frac{\sqrt 2}{2}\\
				\frac{3\sqrt 2}{2}
			\end{pmatrix}$\\
			\begin{multicols}{2}
				$i \leftrightarrow \begin{pmatrix}
				1\\0
				\end{pmatrix}$\\
				$\A i \leftrightarrow \begin{pmatrix}
					\frac{\sqrt 2}{2}\\
					\frac{\sqrt 2}{2}
				\end{pmatrix}$\\
				\includegraphics[width=100px]{6}\\
			\end{multicols}
			\item 
			$\A = \frac{d}{dt}: \underset{1, t, t^2}{p_2} \rightarrow \underset{1, t, t^2}{p_2}\\
			A = \begin{pmatrix}
				0 & 1 & 0\\
				0 & 0 & 2\\
				0 & 0&0
			\end{pmatrix}\\
			(\overbrace{3t^3 + 6t + 4}^{u(t)})' = 6t + 6\\
			3t^2 + 6t + 4 = \begin{pmatrix}
				4\\6\\3
			\end{pmatrix}\\
			\A\mathscr{u} \leftrightarrow \begin{pmatrix}
				0&1&0\\0&0&2\\
				0&0&0
			\end{pmatrix}\cdot
			\begin{pmatrix}
				4\\6\\3
			\end{pmatrix}
			= \begin{pmatrix}
				6\\6\\0
			\end{pmatrix} \leftrightarrow 6 + 6t$
		\end{mylist}
	\end{examples}
	\begin{theorem}[Преобразование матрицы линейного отображения при замене базиса]
		$\A \in L(U, V)$\\
		\begin{tabular}{lll}
			$U$ & $\xi = (\xi_1\ldots\xi_n)$ & -- базисы  $\ \ \ \ \A \xrightarrow{(\xi, \eta)} A$\\
			& $\xi' = (\xi'_1\ldots\xi'_n)$
		\end{tabular}\\
		$T_{\eta\rightarrow\eta'}$ -- матрица перехода\\
		\begin{tabular}{lll}
			$V$ & $\eta = (\eta_1\ldots\eta_m)$ & -- базисы  $\ \ \ \ \A \xrightarrow{(\xi', \eta')} A'$\\
			& $\eta' = (\eta'_1\ldots\eta'_m)$
		\end{tabular}\\
		\mybox{
			$\A' = T^{-1}_{\eta\rightarrow\eta'} \cdot A \cdot T_{\xi\rightarrow\xi'}$	
		}
	\end{theorem}
	Ну видимо сейчас доказательство, но я не уверен.
	\begin{proof}\ \\
		\begin{tabular}{c c c}
 			$\underset{\xi_1\ldots\xi_n}{U}$ & \stackanchor{$\xrightarrow{\A}$}{$\rightharpoonup$} & $\underset{\eta_1\ldots\eta_m}{V}$\\
			$\E_u \uparrow \upharpoonright$ & & $\downharpoonleft \uparrow \E_v$\\
			$\underset{\xi'_1\ldots\xi'_n}{U}$ & $\xrightarrow{\A}$ & $\underset{\eta'_1\ldots\eta'_m}{V}$
		\end{tabular}
		\\\\
		$\A = \E^{-1}_v\A\E_u\leftrightarrow A' = T^{-1}_{\eta\rightarrow\eta'}AT_{\xi\rightarrow\xi'}\\
		\A\B \leftrightarrow AB\\
		\A^{-1}\leftrightarrow A^{-1}\\
		\E^{-1}_v \leftrightarrow T^{-1}_{\eta\rightarrow\eta'}$ Смотри пример 2
	\end{proof}
	\begin{corollary}
		\ \\
		$\A\in End(V) \ \ \ \A: \underset{e_1\ldots e_n}{V} \rightarrow \underset{e_1\ldots e_n}{V}\\
		e_1\ldots e_n$ базис $V \leftrightarrow A\\
		e'_1\ldots e'_n$ базис $\leftrightarrow A'\\
		\A: \underset{e'_1\ldots e'_n}{V} \xrightarrow{A'} \underset{e'_1\ldots e'_n}{V}\\
		T = T_{e\rightarrow e'}$\\
		\mybox{$A' = T^{-1}AT$}
	\end{corollary}
	\begin{remark}
		В условиях теоремы $v = \A u \begin{array}{c}
			\xleftrightarrow{(\xi, \eta)} v = Au\\
			\xleftrightarrow{(\xi', \eta')} v' = A'u
		\end{array}\\
		V = T_{\eta\rightarrow\eta'} V'\\
		U = T_{\xi\rightarrow\xi'} U'\\
		T_{\eta\rightarrow\eta'}v' = AT_{\xi\rightarrow\xi'}u'\\
		v' = \underset{A'}{\boxed{T^{-1}_{\eta\rightarrow\eta'}AT_{\xi\rightarrow\xi'}}u'}
		$
	\end{remark}
	\subsection{Инварианты линейного отображения}
	\textbf{Инвариант - свойство, которое сохраняется при некоторых определенных преобразованиях}\\
	$\mathscr{v} = \A\mathscr{u} \leftrightarrow v = Au$\\
	Форма записи действия линейного отображения на вектор инвариантна относительно замены базиса.\\
	$v' = A'u'$
	\begin{defin}
		$A_{m\times n}$\\
		$ImA = span(A_1, A_2,\ldots A_n) = \{\sum\limits_{i = 1}^n \alpha_i A_i|\alpha_i\in K\} = \\
		\{y = Ax \in \R^m(\mathbb{C}^m)| x\in \R^n(\mathbb{C}^n)\}$\\
		$x = \begin{pmatrix}
			\alpha_1\\\vdots\\\alpha_n
		\end{pmatrix}\\
		rgA = dimImA$ --- ранг матрицы\\
		$KerA = \{x\in\R^n(\mathbb{C}^n)|Ax = \0 \} =$ \{множество решений СЛОУ \} --- ядро матрицы\\
		$dimKerA = n-rgA = defA$ --- дефект матрицы\\
		$\boxed{rgA + defA = n}$ --- аналогично теореме о ранге и дефекте
	\end{defin}
	\begin{theorem}
		$\forall \A \in L(U, V)$\\
		\fbox{
			\parbox{73px}{
				$rg\A = rg A$\\
				$def\A = def A$
			}
		}, \\
		где матрица $A$ -- матрица линейного отображения в некоторых базисах пространств $U$ и $V$.\\
		$rg\A$, $def\A$ инвариантны относительно выбора базиса.
	\end{theorem}
	\begin{proof}
		$\A\leftrightarrow\underset{(\xi, \eta)}{A} \xi = (\xi_1\ldots\xi_n)$ базис $U$\\
		$\eta = (\eta_1\ldots\eta_m)$ базис $V$\\
		$Im\A = span(\A\xi_1\ldots\A\xi_n)\\
		\A\xi_i$\stackanchor{$\leftrightarrow$}{$\cong$}$A_i$\\
		Координатный изоморфизм.\\
		Пусть $rgA = k \Rightarrow k$ столбцов линейно независимы, а остальные -- их линейная комбинация.\\
		По свойствам изоморфизма это означает, то из $\A\xi_1\ldots\A\xi_n \ \  k$ линейно независимые, а остальные -- их линейная комбинация $\Rightarrow rg\A = dimIm\A = k\\$
		\belowbaseline[-10pt]{$\begin{matrix}
			dimU\\
			\parallel\\
			n
		\end{matrix}$
		} = \belowbaseline[-10pt]{
			$\begin{matrix}
				rg\A\\
				\parallel\\
				rgA\\
				\parallel\\
				k
			\end{matrix}$
		} $+ def\A\\
		def\A = n-rgA = n-k = dim$ пространства решений $Ax=0 = defA$ 
	\end{proof}
	\begin{corollary}
		$\A$ изоморфизм $\Leftrightarrow A$ невырожденная ($\exists A^{-1}$), где $A$ матрица в некотором базисе.
	\end{corollary}
	\begin{proof}
		Изоморфизм $\Leftrightarrow$ \stackanchor{$defA = 0$}{$dimU = dimV$} $\Leftrightarrow rgA = n \Leftrightarrow A$ невырожденная. 
	\end{proof}
	\begin{theorem}
		$det\A$ не зависит от выбора базиса пространства $V$ (т.е. является инвариантом относительно выбора базиса). И при этом $det\A = detA$, где $A$ -- матрица оператора $\A$ в некотором базисе.
	\end{theorem}
	\begin{proof}
		$V \ e_1\ldots e_n\\
		det\A = det(\A e_1\ldots\A e_n)\\
		\A e_k = \sum\limits_{i_k=1}^n a_{i_k k}e_{i_k} \xrightarrow{A = (a_{ij})} A_k = \begin{pmatrix}
			a_{1k}\\\vdots\\a_{nk}
		\end{pmatrix} = $
		($det$ $n$-форма, т. е. полиномиальная форма) \\
		$ = \sum\limits_{i_1 = 1}^n\sum\limits_{i_2 = 2}^n\ldots\sum\limits_{i_n = n}^n a_{i_1 1}a_{i_2 2}\ldots a_{i_n n}
		\ det(e_{i_1}, e_{i_2}\ldots e_{i_n}) = $ 
		($n$-форма -- 2 одинаковых аргумента $\Rightarrow det = 0$)$\\
		 = \sum\limits_{\sigma = (i_1\ldots i_n)} a_{i_1 1} a_{i_2 2}\ldots a_{i_n n} 
		 \overbracket{det(\underset{\text{все разные}}{e_{i_1}\ldots e_{i_n}})}^{(-1)^{\E(\sigma)} \ \ det(e_1\ldots e_n) = 1}
		 = \sum\limits_{\sigma = (i_1\ldots i_n)}(-1)^{\E(\sigma)} a_{i_1 1} a_{i_2 2}\ldots a_{i_n n} = det A$\\\\
		 $e'_1\ldots e'_n $ базис $V\\
		 T = T_{e\rightarrow e'}\\
		 det\A = det A' \overset{?}{=} det A\\
		 A' = T^{-1} A T\\
		 det A' = det T^{-1} \cdot detA \cdot det T = det A$
	\end{proof}
	\begin{defin}
		$A, B$ называются подобными, если \\
		$\exists $ невырожденная $C: B = C^{-1} A C$
	\end{defin}
	\begin{examples}
		Матрицы линейного оператора в разных базисах подобны\\
		$A' = T^{-1}AT$\\
		$A, B$ подобны $\Rightarrow det A = det B$
	\end{examples}
	\begin{corollary}
		$f$ -- $n$-форма на $V\\
		\forall \xi_1\ldots\xi_n \ \ \forall \A \in End(V)\\
		\Rightarrow \boxed{f(\A\xi_1\ldots\A\xi_n) = det\A \ f(\xi_1\ldots\xi_n)}$
	\end{corollary}
	\begin{proof}
		$f(\A\xi_1\ldots\A\xi_n) = \\
		\underset{n\text{-форма}}{
		g(\xi_1\ldots\xi_n)} = det(\xi_1\ldots\xi_n)\cdot g(e_1\ldots e_n) = \\
		det(\xi_1\ldots\xi_n)\cdot \underset{\text{смотри док-во теоремы}}{f(\A e_1\ldots \A e_n)} = 
		det(\xi_1\ldots\xi_n)\sum\limits_\sigma (-1)^{\E(\sigma)} a_{i_1 1}\ldots a_{i_n n}\cdot f(e_1\ldots e_n) = \\
		\A e_k = \sum\limits_{k = 1}^n a_{i_k k}e_{i_k} =
		\underbracket{det(\xi_1\ldots\xi_n)f(e_1\ldots e_n)}_{f(\xi_1\ldots\xi_n)}\underbracket{det A}_{det\A}
		$
	\end{proof}
	\begin{remark}
		$A$ -- линейный оператор, $B_{n\times n}\\
		AB = (AB_1 \ AB_2 \ldots AB_n)\\
		det(AB) = det(AB_1 \ldots AB_n) = \\
		= detA\cdot det(B_1\ldots B_n) = detA \cdot det B$
	\end{remark}
	\begin{corollary}
		$\A, \B \in End(V)\\
		det(\A\B) = det\A \cdot \B$
	\end{corollary}
	\begin{proof}
		$det(\A\B) = det(AB) = detA\cdot detB = det\A \cdot det\B$
	\end{proof}
	\begin{corollary}
		$\A\in Aut(V)\\
		\Leftrightarrow det\A \neq 0$\\
		Причем $det\det\A^{-1} = \frac{1}{det\A}$
	\end{corollary}
	\begin{proof}
		Из следствия 2\\
		$\A\A^{-1} = \A^{-1}\A = \E\\
		det\A \cdot det\A^{-1} = det\E = 1 \Rightarrow \ldots$
	\end{proof}
	\begin{examples}$V_3$\\
		\begin{multicols}{2}
		\includegraphics[width=150px]{7}\\	
		$V_{abc \text{--правая тройка}} = \underset{\text{смешанное пр-е}}{\vec{a}\vec{b}\vec{c}} = 
		f(\underset{\text{3-форма}}{\vec{a}\vec{b}\vec{c}})$\\
		$\A\in End(V_3) \ u\in V_3 \rightarrow v = \A u \in V_3$\\
		Как поменяется объем параллелепипеда при линейном преобразовании? 
		\end{multicols}
	$\A (V_{(\vec{a}\vec{b}\vec{c})}) = f(\A\vec{a}, \A\vec{b}, \A\vec{c}) = 
	det\A\cdot f(\vec{a}, \vec{b}, \vec{c}) = det\A \cdot V(\vec{a} \vec{b} \vec{c})\\
	\lambda = |det\A|\ \ \ \ $ Объем увеличится в $\lambda$ раз.\\
	\begin{mylist}
		\item 
		$\A: V_3 \rightarrow V_3$\\
		Оператор подобия\\
		$\forall u\in V_3: \A u = \mu u, \mu \in \R$\\
		\begin{minipage}{0.2\textwidth}
			$A?$
		\end{minipage}
		\begin{minipage}{0.4\textwidth}
			$\A\vec{i} = \mu\vec{i} \leftrightarrow \begin{pmatrix}\mu\\0\\0\end{pmatrix}\\
			\A\vec{j} = \mu\vec{j} \leftrightarrow \begin{pmatrix}0\\\mu\\0\end{pmatrix}\\
			\A\vec{k} = \mu\vec{k} \leftrightarrow \begin{pmatrix}
				0\\0\\\mu
			\end{pmatrix}$
		\end{minipage}
		\begin{minipage}{0.4\textwidth}
			$A = \begin{pmatrix}
				\mu & 0 & 0\\
				0 & \mu & 0\\
				0 & 0 & \mu
			\end{pmatrix}$
		\end{minipage}
	$\lambda = |det\A| = |detA| = |\mu^3|$\newpage
	\item 
	$\A: V_3\rightarrow V_3$\\
	\textbf{Оператор поворота}\\\\
	$\A:  \ \ \ \ \ \begin{matrix}
		\vec{i} \rightarrow e_1 \nearrow \\\vec{j}\rightarrow e_2 \rightarrow\\\vec{k}\rightarrow e_3\searrow
	\end{matrix} \ \ \ 
	\begin{matrix}
		\begin{pmatrix}
			\cos\alpha_1\\
			\cos\beta_1\\
			\cos\gamma_1
		\end{pmatrix}\\
		\begin{pmatrix}
			\cos\alpha_2\\
			\cos\beta_2\\
			\cos\gamma_2
		\end{pmatrix}\\
		\begin{pmatrix}
		\cos\alpha_3\\
		\cos\beta_3\\
		\cos\gamma_3
		\end{pmatrix}\\
	\end{matrix}$\\
	\begin{minipage}{0.5\textwidth}
		\includegraphics[width=0.5\textwidth]{8}
	\end{minipage}
	\begin{minipage}{0.5\textwidth}
		$|e_i| = 1\\
		(e_i, e_j) = 0\\
		i\neq j$
	\end{minipage}\\\\
	"$\A(V_{\vec{a}\vec{b}\vec{c}})$" $= det\A\cdot V_{\vec{a}\vec{b}\vec{c}} = V_{\vec{a}\vec{b}\vec{c}}\\\\
	A = \begin{pmatrix}
		\cos\alpha_1 & \cos\alpha_2 & \cos\alpha_3\\
		\cos\beta_1 & \cos\beta_2 & \cos\beta_3\\
		\cos\gamma_1 & \cos\gamma_2 & \cos\gamma_3\\
	\end{pmatrix}\\\\
	detA = |\cdots| \underset{\text{Смешанное произведение}}{e_1e_2e_3} = 1$\\
	$(detA)^2 = detA\cdot detA^T = det(AA^T) = det\begin{pmatrix}
		(e_1, e_1) & (e_1, e_2) & (e_1, e_3)\\
		(e_2, e_1) & (e_2, e_2) & (e_2, e_3)\\
		(e_3, e_1) & (e_3, e_2) & (e_3, e_3)\\
	\end{pmatrix}
	= detE = 1\\
	|detA| = 1$
	\end{mylist}
	\end{examples}
	\begin{stat}
		$A, B$ подобные матрицы $\Rightarrow trA = trB$\\
		$trace = $ след
	\end{stat}
	\begin{proof}
		$A, B$ подобные $\Rightarrow\\
		\exists\ C$ невырожденная$: C^{-1}(AC) = B$\\
		$$trB = \sum\limits_{i=1}^n b_{ii} = \sum\limits_{i = 1}^n \sum\limits_{j=1}^n C^{\text{''}-1\text{''}}_{ij}(AC){ji} = 
		\sum\limits_{i = 1}^n \sum\limits_{j=1}^n \sum\limits_{k = 1} ^ n C^{''-1''}_{ij} a_{jk} C_{ki} = 
		\sum_{j=1}^n\sum_{k=1}^n a_{jk} \underbracket{\sum_{i=1}^n C_{ki} C^{''-1''}_{ij}}_{\delta_{kj}} = \sum_{k=1}^n a_{kk} = trA 
		$$
		$\boxed{\delta_{kj} = \left[\begin{matrix}1, k=j\\0, k\neq j \end{matrix} \right.} \ \ CC^{-1} = E$
	\end{proof}
	\begin{defin}
		$tr\A = trA$, где $A$ -- матрица оператора в некотором базисе.\\
		$tr\A = trA = trA'$ --- не зависит от выбора базиса, т.к. $A$ и $A'$ подобны.
	\end{defin}
	\begin{defin}
		$L\subset V \ L$ инвариантно относительно $\A\in End(V)$
		если $\forall u\in L: \A u\in L$
	\end{defin}
	\begin{examples}\ \\
		\begin{mylist}
			\item $\0, V$ инвариантны относительно $\A$
			\item $Ker\A, Im\A $ инвариантны относительно $\A$\\
			\begin{minipage}{0.3\textwidth}
				\includegraphics{9}
			\end{minipage}
			\begin{minipage}{0.7\textwidth}
				\begin{flushleft}
					$\A:V_3\rightarrow V_3$\\
					Поворот вектора(пр-ва) относительно оси $l$ на угол $\alpha$
				\end{flushleft}
			\end{minipage}\\
			\begin{minipage}{0.3\textwidth}
				\includegraphics{10}
			\end{minipage}
			\begin{minipage}{0.7\textwidth}
				\begin{flushleft}
					Плоскость $\perp l$ инвариантна относительно $\A$\\
					$P = x_0 + L$ инвариантно
				\end{flushleft}
			\end{minipage}
		\end{mylist}
	\end{examples}
	\begin{theorem}
		$L\subset B \ \ \A\in End(V)$. Линейное пространство инвариантно относительно $\A$\\
		$\Rightarrow \exists$ базис пространства $V$, т.ч. матрица оператора $\A$ в этом базисе\\ будет иметь вид: $A = \left(\begin{array}{c|c}A_1 & A_2\\
		\hline
		\0 & A_3\end{array}\right)\\
		A_1 k\times k$ где $k = dimL$
	\end{theorem}
	\begin{proof}
		$L = span(\underset{\text{базис}}{e_1\ldots e_k})$\\
		Дополним до базиса $V: e_1\ldots e_k e_{k+1}\ldots e_n\\
		e_i \in L \Rightarrow \underset{1\leq i \leq k}{\A e_i}\in L = \sum\limits_{m=1}^k a_{mi}e_m + 
		\sum\limits_{m = k+1}^n 0\cdot e_m
		\leftrightarrow A_i^1 = \begin{pmatrix}
			a_{1i}\\\vdots\\ a_{ki}\\0\\\vdots\\0
		\end{pmatrix}\\
		\underset{k+1\leq i \leq n}{\A e_i} = \sum\limits_{j=1}^n a_{ij} e_j \leftrightarrow 
		A^{2, 3}_i = \begin{pmatrix}
			a_{1i}\\\vdots\\a_{ni}
		\end{pmatrix}
		\Rightarrow A = \begin{pmatrix}
			\boxed{\begin{matrix}
				a_{1i} & \\\vdots & A_i^1 \\a_{ki} & 
				\end{matrix}
			} & \boxed{
				\begin{matrix}
					A^{2, 3}_i\\\ldots
				\end{matrix}
			}\\
			0 & \boxed{\begin{matrix}
			\ldots\\\ldots
			\end{matrix}}
		\end{pmatrix}$
	\end{proof}
	\begin{corollary}
		$V = \bigoplus\limits_{i=1}^m L_i \ \ \ L_i$ инвариантно $\A\\
		\Rightarrow \exists \ $ базис пр-ва $V$, в котором матрица оператора $\A$ будет иметь \underline{блочно-диагональный вид:}\\
		$A = \begin{pmatrix}
			\boxed{A^1} & \ldots & 0\\
			& \boxed{A^2} & \\
			0 & & \boxed{A^n}
		\end{pmatrix}\\
		\left(\underset{\text{размерность матрицы}}{A^i}\right) = dimL_i$
	\end{corollary}
	\begin{proof}
		$L_1 = span(\underset{\text{базис}}{e^1_i\ldots e^{i_k}_i})$\\
		т.к. $\bigoplus$, то базис $V$ -- объединение базисов $L_i\\
		V = span(e^1_1\ldots e^{i_m}_m)\\
		\A^j e_i \in L_i \Rightarrow$ раскладываем по базису $L_i \Rightarrow$\\ на остальных позициях в столбике матрицы оператора будут нули.\\
		$A = \left(
		\begin{array}{c|c|c}
			\overset{\stackrel{L_1}{\underline{1\ldots i_1}}}{\begin{matrix}* & * &*\\ *& * & *\\
				* & * & *\end{matrix}} & 
			\overset{\stackrel{L_2}{\underline{i_1 + 1\ldots i_2}}}
			{\begin{matrix}0 & 0 & 0\\ 0 & 0& 0\\0&0&0\end{matrix}} &
			\begin{matrix}
				0\\0\\0
			\end{matrix} \\
			\hline 
			\begin{matrix}0 & 0 & 0\\ 0 & 0& 0\\0&0&0\end{matrix} 
			& \begin{matrix}* & * &*\\ *& * & *\\
			* & * & *\end{matrix} &
			\left.\begin{matrix}*\\\vdots\\ *\end{matrix}\right\}\\
			\hline
			\begin{matrix}0 & 0 & 0\\ 0 & 0& 0\\0&0&0 \end{matrix}& \begin{matrix}
				0 & 0 & 0\\ 0 & 0& 0\\0&0&0
			\end{matrix} & \begin{matrix}0\\0\\0\end{matrix}
		\end{array}\right) \text{отвечает позиции базисных элементов 
			пр-ва }L_i\text{в базисе }V$
	\end{proof}
	\begin{corollary}
		$V = \bigoplus\limits_{i=1}^m L_i \ \ \ \ L_i$ инвариантно относительно $\A\\
		\A\in End(V)
		\Rightarrow V = \bigoplus\limits_{i=1}^m Im \A|_{L_i}$
	\end{corollary}\newpage
	\begin{proof}
		$V = \bigoplus\limits_{i=1}^m L_i \Rightarrow \forall\ u \in V \ \  \exists! u = \sum\limits_{i = 1}^m u_i \in L_i\\
		Im\A \subset \sum\limits_{i = 1}^m Im\A|_{L_i}\\
		v\in Im\A = \A u = \sum\limits_{i = 1}^m \A u_i \in Im\A|_{L_i}\\
		$
		\textbf{Верно и "$\supset$"}\\
		Пусть $v_i \in Im\A|_{L_i} : v_i = \A u_i, u_i\in L_i\\
		\sum\limits_{i = 1}^m v_i = \sum\limits_{i = 1}^m \A u_i = \A(\sum\limits_{i = 1}^m u_i \in V)\in Im\A\\
		Im\A = \sum\limits_{i = 1}^m Im\A|_{L_i}\\
		\bigoplus$ прямая?\\
		$v_i\in Im\A|_{L_i}\\
		v_i = \A u_i \ \ \ \ u_i \in L_i\\
		\sum\limits_{i = 1}^m v_i = \0 \longleftarrow\\
		\text{Т.к. }L_i $ инвариантна $\Rightarrow \A u_i \in L_i \Rightarrow v_i \in L_i$, но $L_i$ дизъюнктны $\nwarrow \Rightarrow \forall i : v_i = \0\\
		\Rightarrow Im \A|_{L_i}$ дизъюнктны $\Rightarrow \bigoplus$
	\end{proof}
	\subsection{Собственные числа и собственные вектора линейного оператора.}
	$\A\in End(v) \ \ \ \ V$ линейное пространство над $K$
	\begin{defin}
		$\lambda\in K \text{ -- \textbf{собственное число} (c.ч.) линейного оператора }\A$, если $\\
		\exists \ \boxed {v\in V \neq \0}$, который называется \textbf{собственным вектором} (с.в.), такой что $\boxed{\A v = \lambda v}$
	\end{defin}
	Пусть $v: \A v = \lambda v \Leftrightarrow (\A-\lambda\E)v = 0 \Leftrightarrow v\in Ker(\A-\lambda\E)$
	\begin{defin}
		$V_\lambda = Ker(\A-\lambda\E) = \{\text{с.в. }v \text{ и }\0\}$ называется собственным подпространством.\\
		$\boxed{\gamma(\lambda):= \dim V_\lambda} $ -- геометрическая кратность с.ч.
	\end{defin}
	$\gamma\geq 1\\
	V_\lambda$ и $\gamma(\lambda)$ -- инварианты относительно выбора базиса.\\
	$v\in V_\lambda \ \ \A v = \lambda v\overset{?}{\in} V_\lambda\\
	\A(\lambda v) = \lambda\A v = \lambda^2 v = \lambda(\lambda v)$
	\begin{examples}
		\ \\
		\begin{mylist}
			\item 
			$\A$ -- оператор подобия:\\
			$\A v = \mu \cdot v \ \ \ \mu \in K\\
			\mu$ с.ч. $\ \ V_\lambda = V$
			\item 
			$\A$ -- оператор поворота на плоскости на угол $\alpha$\\
			\begin{minipage}{0.2\textwidth}
				\includegraphics[width=100px]{11}
			\end{minipage}
			\begin{minipage}{0.8\textwidth}
				$\alpha \neq \pi k \Rightarrow$ нет с.в.
			\end{minipage} \\
			\item 
			Пусть $\lambda$ с.ч.$=0 \ \ \ \A v = \0 \ $с.в. $\neq \0 \ \ \Leftrightarrow\\
			\Leftrightarrow Ker\A$ нетривиально $\Leftrightarrow \A$ не автоморфизм $\Leftrightarrow \A$ необратимо $\Leftrightarrow det\A = 0$ 
			\item $\A: V\rightarrow V\\
			v_1\ldots v_n$ базис, т.ч. $A = \begin{pmatrix}
				\lambda_1 & \ldots & 0\\
				\ldots & \ldots & \ldots\\
				0 & \ldots & \lambda_n
			\end{pmatrix} = diag(\lambda_1\ldots\lambda_n) = \Lambda$\\
			Базис состоит из с.в. отвечающих с.ч. $\lambda_1 \ldots \lambda_n\\
			\A v_i = \lambda_i v_i \ \ \ A_i = \begin{pmatrix}
				0\\0\\\lambda_i\\\vdots\\0
			\end{pmatrix}\\
			\lambda$ -- с.ч. $v$ с.в. $\neq \0 \Leftrightarrow Ker(\A-\lambda\E)$ нетривиально $\Leftrightarrow det(\A - \lambda\E) = 0$
		\end{mylist}
	\end{examples}
	\begin{defin}
		$\chi_\A (t) = det(\A-t\E)$ -- характеристический многочлен оператора $\A, t\in K$
	\end{defin}

		$V e_1\ldots e_n $ базис $\ \ \A\leftrightarrow A\\
		\chi_\A(t) = det(\A-t\E) = det(A-tE)$ т.к. $det$ оператора инвариантен относительно выбора базиса.\\
		$\chi_\A(t) = det(A-tE) = \left|\begin{matrix}
			(a_{11}-t) & a_{12} & \ldots & a_{1n}\\
			a_{21} & (a_{22}-t) & \ldots & \ldots\\
			\ldots & \ldots & \ldots & \ldots\\
			a_{n1} & \ldots & \ldots & (a_{nn}-t)
		\end{matrix}\right| = \\
		= (-1)^n t^n + (-1)^{n-1}(\underset{tr A = tr\A}{a_{11} + \ldots + a_{nn}})t^{n-1} + \ldots + \underset{det\A}{detA}$\\\\
		По теореме Виета: $det\A = \underset{\text{корни }\chi_\A(t)}{\lambda_1 \ldots \lambda_n}\\\\
		\underline{\underline{\lambda\in K}}$ с.ч. $\Leftrightarrow \chi_\A(\lambda) = 0 \ \ (\underline{\underline{\lambda\in K}})$\\
		$\lambda$ корень характеристического многочлена.\\
		$k = \mathbb{C} \Rightarrow n$ с.ч. с учетом кратности корней характеристического многочлена.\\
		$k = \R \Rightarrow$ только вещественные корни $\chi_A$ будут с.ч.\\
		$\chi_\A(t) = (-1)^n \prod\limits_{\lambda \text{ корень}} (t-\lambda)^{\alpha(\lambda)}\\
			\alpha(\lambda)$ называется алгебраической кратностью с.ч. $\lambda$ (если $\lambda \in K$)
	\begin{defin}
		Множество всех с.ч. с учетом алгебраической кратности называется \textbf{спектром} линейного оператора. ($\lambda, \alpha(\lambda)$)\\
		Спектр -- простой, если все с.ч. попарно-различны.\\
		$\alpha(\lambda) = 1 \ \forall \ \lambda$
	\end{defin}
	\textbf{Немножко про алгебраическую кратность}\\
	$f(t) = a_nt^n + a_{n-1}t^{n-1} + \ldots + a_1t + a_0 = a_n\prod\limits_{a \text{--корень}}(t-a)^{m_a}\\
	a\text{--корень } f \Leftrightarrow f(a) = 0 \Leftrightarrow f\ \vdots\ (t-a)\\
	a$ -- корень $f$ \textbf{кратности} $m \Leftrightarrow \begin{matrix}
		f \mid (t-a)^m\\
		f \nmid (t-a)^{m+1}
	\end{matrix}\\
	\Leftrightarrow f(t) = (t-a)^m g(t)\\
	a_0$ -- произведение всех корней с учетом кратности = $(-1)^n \prod a \ \ \ \ \ a$--корень с учетом кратности
	\\\\
	$\det\A = \lambda_1 \ldots \lambda_n\\
	(-1)^n t^n + \ldots = (-1)^n (t-a_1)(t-a_2)\ldots(t-a_n)\\
	\chi_\A (t) = (-1)^n t^n + \ldots = (-1)^n (t-\lambda_1) \ldots (t-\lambda_n)\\
	\det\A = \lambda_1\ldots\lambda_n = 0 \Leftrightarrow \lambda = 0 $ с.ч.
	\begin{examples}
		$\A$ -- поворот на угол $\alpha$\\
		\begin{minipage}{0.3\textwidth}
			\includegraphics[width=\textwidth]{11}
		\end{minipage}
		\begin{minipage}{0.7\textwidth}
			$\vec{i}\ \vec{j}\\
			\A \leftrightarrow A = \begin{pmatrix}
				\cos\alpha & -\sin\alpha\\
				\sin\alpha & \cos\alpha
			\end{pmatrix}$
		\end{minipage}
		$\chi_\A(t) = det \begin{pmatrix}
			\cos\alpha-t & -\sin\alpha\\
			\sin\alpha & \cos\alpha-t
		\end{pmatrix} = \\
		\cos^2\alpha - 2\cos\alpha t + t^2 + sin^2\alpha = t^2 - 2\cos\alpha t + 1\\
		D = 4\cos^2\alpha - 4 < 0 \ \ \ \alpha\neq \pi k$\\
		нет вещ. корней $\Rightarrow$ нет с.ч.\\
		$K = \R$
	\end{examples}
	\begin{theorem}
		$\lambda$ с.ч. $\A \Rightarrow \boxed{1\leq \gamma(\lambda) \leq \alpha (\lambda)}$
	\end{theorem}
	\begin{proof}
		Пусть $\gamma(\lambda) = k = dimV_\lambda = span(\underset{\text{базис}}{v_1\ldots v_k})\\
		V_\lambda$ инвариантно относительно $\A\Rightarrow \exists $ базис: матрица оператора будет иметь вид:\\
		\textit{(инвариантное линейное подпространство. Смотри Теорему пункта 7.3}\\
		$A = \begin{pmatrix}
			A^1 & \vline & A^2\\
			\hline
			0  & \vline & A^3	
		\end{pmatrix}
		= \left(\begin{array}{c|c}
			\begin{array}{cc}
				\lambda & 0\\
				0 & \lambda
			\end{array} & A^2\\
			\hline
			0  & A^3
		\end{array}\right) \ \ A^1_{k\times k}$\\
		
		Базис = $v_1\ldots v_k v_{k+1} \ldots v_n\\
		\A \underset{i = 1\ldots k}{v_i} \in V_\lambda = \lambda v_i \leftrightarrow A^1_i = \begin{pmatrix}0\\0\\\lambda\\0\\0\\0\end{pmatrix}$\\
		$\chi_\A(t) = det \left(
		\begin{array}{c|c}
			\begin{matrix}
				\lambda-t & 0\\0 & \lambda - t
			\end{matrix} & A^2\\
			\hline
			0 & A^3 - tE_{n-k}
		\end{array}\right) \underset{\text{св-ва } det}{=} \begin{vmatrix}
			\lambda - t & 0\\0 & \lambda-t
		\end{vmatrix}|A^3-t E_{n-k}| = (\lambda-t)^k\chi_{\A^3}(t)$\\
		Очевидно, $\lambda$ корень $\chi_\A(t)$ кратности не меньше, чем $k \Rightarrow\alpha(\lambda) \geq k = \gamma(\lambda)$
	\end{proof}
	\begin{theorem}
		$\lambda_1\ldots\lambda_m $ -- различные с.ч. $\A\\
		v_1\ldots v_m$ соответствующие им с.в. $\Rightarrow\\
		\Rightarrow v_1\ldots v_m$ линейно независимы.
	\end{theorem}
	\begin{proof} Метод математической индукции
		\begin{mylist}
			\item База. $m=1 \ \ \ \lambda_1 v_1$ с.в. -- линейно независимы, т.к. $v_1\neq \0$
			\item Индукционное предположение. Пусть верно для $m-1$
			\item Индукционный переход. Докажем, что верно для $m$\\
			От противного. Пусть $\lambda_1\ldots \lambda_m $ попарно различные с.ч. $\A$,\\
			а $v_1\ldots v_m$ линейно зависимы.\\
			Пусть $v_m = \sum\limits_{i=1}^{m}\alpha_i v_i\\
			\begin{array}{l}
				\A_{v_m} = \sum\limits_{i=1}^{m-1}\alpha_i \A_{v_i} = \sum\limits_{i=1}^{m-1} \alpha_i \lambda_i v_i\\
				
					||
				\\
				\lambda_m v_m = \sum\limits_{i=1}^{m-1} \alpha_i \lambda_m v_i
			\end{array}$\\
			$\sum\limits_{i=1}^{m-1}\alpha_i(\underset{\stackrel{0}{\nparallel}}{\lambda_i - \lambda_m})v_i = \0 \ \ v_i$ линейно независим по инд. предположению\\
			$\Leftrightarrow \alpha_i = 0 \ \ \forall \ i=1\ldots m-1 \Rightarrow\\
			\Rightarrow v_m = \0$ --- Противоречие, т.к. $v_m$ с.в. и значит не может быть $\0$
		\end{mylist}
	\end{proof}
	\begin{corollary}
		$\lambda_1\ldots\lambda_m$ различные с.ч. $\A
		\Rightarrow  V_{\lambda_1}\ldots V_{\lambda_m}$ дизъюнктны.
		$\left(\bigoplus\limits_{\stackrel{\lambda}{\text{с.ч.}}} V_\lambda\right)$ 
	\end{corollary}
	\begin{proof}
		$v_1 + \ldots +v_m = \0 \ \ v_i\ \in V_{\lambda_i}$\\
		Если хотя бы 1 слагаемое $\neq \0 \Rightarrow$ это слагаемое с.в. $\Rightarrow$ противоречие с линейной независимостью с.в., отвечающих различным с.ч. $\Rightarrow \forall \ i: v_i = \0 \Rightarrow$ дизъюнктны. 
	\end{proof}
	\begin{theorem}
		$V = \bigoplus\limits_{i=1}^m L_i \ \ L_i$ инвариантно относительно $\A\\
		\A_i = \A|_{L_i} : L_i \rightarrow L_i
		\Rightarrow \boxed{\chi_\A(t) = \prod_{i=1}^{m}\chi_{\A_i}(t)}$
	\end{theorem}
	\begin{proof}
		см. теорему - следствие п. 7.3\\
		Базис $V$ -- объединение базисов $L_i\\
		\A \leftrightarrow A = \begin{pmatrix}
			\boxed{A^1} & & 0\\
						& \boxed{A^2} & \\
			0 & & \boxed{A^m}
		\end{pmatrix}\\
		\A_i \leftrightarrow A^i \ \ \ \ \; A_{k_i\times k_i}\\
		\chi_\A(t) = |A-tE| \underset{\text{свойства }det}{=} |A^1-tE_{k_1}||A^2-tE_{k_2}|\ldots|A^m-tE_{k_m}| = \\
		\begin{matrix}
			\chi_{A^1}(t) & \chi_{A^2}(t) & \ldots & \chi_{A^m}(t)\\
			|| & || & & ||\\
			\A_1 & \A_2 & & \A^m
		\end{matrix}
	$
	\end{proof}
	Все свойства с.ч. и с.в. доказанные для оператора верны для числовых матриц пространств $\R^m, \mathbb{C}^m$.\\
	$A_{n\times n} \  \ \; \lambda$ с.ч. $A: \exists x\in \R^n \neq \0 \ \; \ Ax = \lambda x\\
	y = \underset{\stackrel{\uparrow}{\text{линейный оператор}}}{Ax}$
	\begin{examples}
		$A = \begin{pmatrix}
			4 & -5 & 2\\
			5 & -7 & 3\\
			6 & -9 & 4
		\end{pmatrix}$\\
		с.ч., с.в.?  $\alpha(\lambda), \gamma(\lambda)$?\\
		$\chi_{\A}(t) = \chi(t) = \begin{vmatrix}
			4-t & -5 & 2\\
			5 & -7-t & 3\\
			6 & -9 & 4-t
		\end{vmatrix} = \begin{vmatrix}
			4-t & 1-t & 2\\
			5 & 1-t & 3\\
			6 & 1-t & 4-t
		\end{vmatrix} = (1-t)\begin{vmatrix}
			4-t & 1 & 2\\
			5 & 1 & 3\\
			6 & 1 & 4-t
		\end{vmatrix}
		= (1-t)t^2\\
		t_1 = 0 \; \;\alpha(0) = 2\\
		t_2 = 1 \; \;\alpha(1) = 1\\
		\\
		V_\lambda = Ker(A-\lambda E) \ \ \ \ \ A = \begin{pmatrix}
			4 & -5 & 2\\
			5 & -7 & 3\\
			6 & -9 & 4
		\end{pmatrix}\\
		\lambda_1 = 0 \ \ \ \begin{pmatrix}
			4 & -5 & 2 \ \vline & 0\\
			5 & -7 & 3 \ \vline & 0\\
			6 & -9 & 4 \ \vline & 0
		\end{pmatrix} \sim \ldots \begin{pmatrix}
			x_1\\x_2\\x_3
		\end{pmatrix} = \alpha\begin{pmatrix}
		1\\2\\3
		\end{pmatrix}\ \ \ \alpha\in]R\\
		V_{\lambda_1} = 0 = span\begin{pmatrix}
		1\\2\\3
		\end{pmatrix}\\
		\gamma(0) = 1 < \alpha(0)\\
		\lambda_2 \ \ 1\leq\gamma \leq\alpha = 1\\
		\begin{pmatrix}
		3 & -5 & 2 \ \vline & 0\\
		5 & -8 & 3 \ \vline & 0\\
		6 & -9 & 3 \ \vline & 0
		\end{pmatrix} \sim \ldots \begin{pmatrix}
		x_1\\x_2\\x_3
		\end{pmatrix} = \alpha\begin{pmatrix}
		1\\1\\1
		\end{pmatrix}
		\ \ \alpha\in\R\\
		V_{\lambda_2} = span\begin{pmatrix}
		1\\1\\1
		\end{pmatrix}\\
		\gamma(1) = 1
		$
	\end{examples}
	\subsection{Оператор простой структуры. (о.п.с.) \\
		Проекторы. Спектральное разложение о.п.с.
		\\ Функция от матрицы.}
	\begin{defin}
		$\A\in End(V)\\$
		$\A$ называется о.п.с., если $\exists$ базис пространтсва $V$, т.ч. матрица оператора в этом базисе имеет диагональный вид $\Lambda = diag(\lambda_1\ldots\lambda_n) = \begin{pmatrix}
			\lambda_1 & 0\\ 0 & \lambda_n
		\end{pmatrix} \Leftrightarrow \exists$ базис $V$ из с.ч. $\A \Leftrightarrow V = \bigoplus\limits_{\lambda \text{с.ч. } \A} V_\lambda\\
		V = span(v_1\ldots v_n)$
	\end{defin}
	\begin{theorem}
		Пусть $\sum\limits_{\lambda \text{с.ч. }\A} \alpha(\lambda) = n = dim V\\
		\Leftrightarrow $все корни $\chi(t) \in K \Leftrightarrow$ все корни $\chi(t)$ являются с.ч. $\A$\\
		$\boxed{\A\text{о.п.с.}\Leftrightarrow \forall\text{с.ч.} \lambda \ \ \ 1 \leq \gamma(\lambda) = \alpha(\lambda)}$
	\end{theorem}
	\begin{proof}
		$\A$ о.п.с. $\Leftrightarrow V = \bigoplus\limits_{\lambda \text{с.ч.}}V_\lambda \Leftrightarrow \\
		\Leftrightarrow n = dim V = \sum\limits_{\lambda \text{с.ч.}}\gamma(\lambda) \underset{\nearrow}{=} \sum\limits_{\lambda\text{с.ч.}}\alpha(\lambda)\\
		1\leq \gamma(\lambda) \leq \alpha(\lambda) \ \ \ \ \; \; \; \nearrow\\
		\sum\limits_{\lambda\text{с.ч.}}\alpha(\lambda) = n\ \  \rightarrow \ \ \nearrow
		\Rightarrow \forall\lambda: \boxed{\gamma(\lambda) = \alpha(\lambda)}$
	\end{proof}
	\begin{corollary}
		$\sum\limits_{\lambda\text{с.ч.}}\alpha(\lambda) = n = dim V\\
		\A $о.п.с. $\Leftarrow $ спектр -- простой.\\
		($n$ попарно различных с.ч. $\forall \lambda \gamma(\lambda) = \alpha(\lambda) = 1$)
	\end{corollary}
	\begin{defin}
		$A_{n\times m}$ называется диагонализируемой, если $\exists$ невырожденная $T_{n\times n}$, т.ч.\\ $T^{-1}AT = \Lambda = diag(\lambda_1\ldots\lambda_n)$\\
		("$A$ подобна диагональной матрице")
	\end{defin}
	\begin{corollary}
		Если матрица $A_{n\times n}$ -- матрица некоторого о.п.с. $\A$, то она \textbf{диагонализируема}. И обратно, любая диагонализируемая матрица является матрицей о.п.с. в некотором базисе.
	\end{corollary}
	\begin{proof}\ \\
		\begin{tabular}{clccc}
			$\A$ & о.п.с. & $\Leftrightarrow$ & $\exists$ базис & $\underset{\text{с.в.}}{v_1\ldots v_n}$\\
			$\updownarrow$ & $\underset{\text{базис}}{(e_1\ldots e_n)}V$ & & & $\underset{\text{с.ч.}}{\lambda_1\ldots \lambda_n}$\\
			$A$ & & & & $\updownarrow$\\
			& & & &$\Lambda = \begin{pmatrix}
				\lambda_1 & 0\\ 0 & \lambda_n
			\end{pmatrix}$
		\end{tabular}\\
		$T = T_{e\rightarrow \mathscr{v}}$ невырожденная.\\
		$\Lambda = T^{-1} A T\\
		A = T \Lambda T^{-1}$
	\end{proof}
	$\boxed{
		\begin{array}{r}
		A \text{ диагонализируема }\Leftrightarrow \sum\limits_{\lambda\text{с.ч.}}\alpha(\lambda) = n\\
		\forall \ \lambda \text{ с.ч.} \ \gamma(\lambda) = \alpha (\lambda)
		\end{array}
	}$
	\begin{defin}\ \\
		$\begin{array}{ccc}
			V = \bigoplus\limits_{i=1}^m L_i & & \mathscr{p}_i : V\rightarrow L_i \subset V\\
			\nwarrow\Leftarrow & \Leftrightarrow & \Rightarrow\searrow\\
			\underset{\text{линейное подпр.}}{L_i \subset V} & & \forall v \in V \  \exists ! : v = \sum\limits_{i=1}^m v_i \in L_i
		\end{array}$\\
		$\boxed{\forall \ v\in V \ \ \ \p_i v \overset{def}{:=}v_i} \ \ \ \ \ \ \; i = 1\ldots m$
	\end{defin}
	\textbf{Оператор проектирования (проектор)}\\
	$\p_i \overset{?}{\in} End(V)\\
	\p_i (u + \lambda v) = u_i + \lambda v_i = \p_i u + \lambda \p_i v \ \ \ \Rightarrow \ \ \ \p_i $ линейный оператор.\\
	$u + \lambda V = \sum\limits_{i=1}^m u_i \in L_i + \lambda \sum\limits_{i=1}^m v_i \in L_i = \sum\limits_{i=1}^m (\underbrace{u_i + \lambda v_i}_{\in L_i})\\
	u_i = \p_i u \ \ \ \ v_i = \p_i v
	$
	\newpage
	\textbf{Свойства проекторов:}
	\begin{mylist}
		\item $\forall \ i \neq j \ \ \p_i \p_ij = \0$
		\item $
		\forall i: \p_i^2 = \p_i \ \ (\Rightarrow \forall k\in \mathbb{N} \o_i^k = \p_i)
		$
		\item $\sum\limits_{i=1}^m \p_i = \E$
		\item $Ker\p_i = \sum\limits_{j\neq i} L_j \ \ \forall i = 1\ldots m\\
		Im\p_i = L_i$
	\end{mylist}
	\begin{proof}\ \\
		\begin{mylist}
			\item $\forall v \in V \ \ \p_i\p_ij(v) = \p_i v_j \in L_j = \0 \Rightarrow \p_i\p_ij = \0$\\
			Т.к. $L_i$ дизъюнктны \\
			$v = v_1 + v_i + \underset{\text{Ед. образом}}{v_j} + \ldots + v_n\\
			v_j = v_j + \0$
			\item $\forall v\in V \ \ \p_i \underbracket{\p_i(v)}_{v_i\in L_i}=v_i = \p_i v$\\
			Т.к. верно $\forall v\in V$, то верно и для базиса $\Rightarrow$ операторы совпадают. $\p_i\p_i = \p_i$
			\item $\forall v \in V (\sum\limits_{i=1}^m \p_i)v = \sum\limits_{i=1}^m \p_i v = \sum\limits_{i=1}^m v_i = v = \E v \Rightarrow \ldots \Rightarrow \sum\limits_{i=1}^m = \E$
			\item 
			$\p_i(\underset{\mathlarger{= \sum\limits_{j \neq i} \underbrace{\p_i v_j}_{\0}}}{v_1 + \ldots + v_{i-1} + v_{i+1} + \ldots + v_m}) + \0\\
			\boxed{
				\begin{matrix}
				\sum\limits_{j\neq i} L_j \subset Ker\ \p_i\\
				\text{т.к. }v = \bigoplus\limits_{j\neq i} L_j \oplus L_i
				\end{matrix}
			} \Rightarrow Ker\ \p_i = \bigoplus\limits_{j\neq i} L_j\\\\
			Im \ \p_i = L_i \text{ по }def \ \ "\subset"$\\
			Верно "$\supset$" $\ \forall v_i \in L_i \leadsto v_i \in V = \p v_i = v_i$
		\end{mylist}
	\end{proof}
	\begin{stat}
		$\underset{i=1\ldots m}{\p_i \in End(V)} : V\rightarrow V$ и выполнены свойства 1, 3 $\Rightarrow\\
		\Rightarrow V = \bigoplus\limits_{i=1}^m Im\p_i $  (т.е. $\p_i$ проекторы на $L_i = Im\p_i$)
	\end{stat}
	\begin{proof}
		\ \\
		\begin{mylist}
			\item Если выполнены 1, 3, то верно 2\\
			$\p_i \p_i \overset{?}{=}\p_i\\
			\p_i = \p_i\E = p_i \sum\limits_{j=1}^m \p_j = \sum\limits_{j=1}^m \underset{\begin{matrix}
				\stackrel{||}{\stackrel{\0}{i\neq j}}
				\end{matrix}}{\p_i\p_j} = \p_i^2$
			\item 
			$v_1 + v_2 + \ldots + v_m = \0\\
			v_i \in Im\p_i \ $ дизъюнктно?\\
			$v_i = \p_i w_i \ w_i\in V\\
			v_i = \p_i w_i \underset{\uparrow}{=} \p_i (\underbrace{\sum\limits_{j=1}^m \underbracket{\p_j w_j}_{v_j}}_{=\0}) = \0\\
			\sum\limits_{j=1}^m \underbrace{\p_i ( p_j}_{=\0 \ i\neq j} w_j) = \p_i^2 w_i = \p_i w_i\\
			\forall \ v\in V \ \E v = v = \sum\limits_{j=1}^m \underset{\mathlarger{\stackrel{||}{v_j \in Im \p_j}}}{\p_j v}
			\Rightarrow v = \sum\limits_{j=1}^m Im \p_j$
		\end{mylist}
	\end{proof}
	\begin{theorem}[О спектральном разложении о.п.с.]
		$v = \bigoplus\limits_{\lambda\text{с.ч.}}V_\lambda \ \ \ \ \ \underset{\text{проекторы}}{\p_\lambda: 
		V \rightarrow V_\lambda}\\
		\A \text{ о.п.с.} \Leftrightarrow \A = \sum\limits_{\lambda\text{с.ч.}}\lambda\p_\lambda \leftarrow $спектральные проекторы
	\end{theorem}
	\begin{proof}
		\ \\
		\begin{mylist}
			\item 
			$\p_\lambda\p_\mu = \0$
			\item $\p^2_\lambda = \p_\lambda$
			\item $\sum\limits_{\lambda \text{с.ч.}}\p_\lambda = \E$
		\end{mylist}
		$\forall v\in V\\
		\A v \underset{\stackrel{\uparrow}{V = \bigoplus\limits_{\lambda} V_\lambda}}{=}
		\A(\sum\limits_{\lambda}v_\lambda\in V_\lambda) = \sum\limits_{\lambda\text{с.ч.}}\underbracket{\A v_\lambda}_{= \lambda v_\lambda} = \\
		\sum\limits_{\lambda\text{с.ч.}}\lambda v_\lambda = \sum\limits_{\lambda\text{с.ч.}}\lambda \p_\lambda v$\\
		Доказательство верно $\forall$ векторного про-ва $V$. В частности для базиса $\Rightarrow \boxed{\A = \sum\limits_{\lambda \text{с.ч.}}\lambda \p_\lambda}$
	\end{proof}
	\begin{corollary}
		\textbf{$A_{n\times n}$ диагонализируема} $\Leftrightarrow$ 
			\belowbaseline[-10pt]{$\begin{array}{c}
				\exists \ \underset{\text{проекторы}}{\p_{\lambda \ n\times n}} \ \ \ 1^\circ\ 2^\circ\ 3^\circ\\
				A = \sum\limits_{\lambda\text{с.ч.}}\lambda \p_\lambda
			\end{array}$}
	\end{corollary}
	\begin{examples}
		$A = $\belowbaseline[-10pt]{$\begin{pmatrix}
				7 &  -12 & 6\\
				10 & -19 & 10\\
				12 & -23 & 13
			\end{pmatrix}$}\\
		$\lambda_1 = 1 \ \alpha(\lambda_1) = \gamma(\alpha_1) = 2\\
		V_{\lambda_1} = span\begin{pmatrix}
			2\\1\\0
		\end{pmatrix}\begin{pmatrix}
			1\\0\\-1
		\end{pmatrix} = span(v_1, v_2)\\
		\lambda_2 = -1 \ \alpha(\lambda_2) = \gamma(\lambda
		_2) = 1\\
		V_{\lambda_2} = span \begin{pmatrix}
			3\\5\\6
		\end{pmatrix} = span \ V_3\\
		\Rightarrow \text{о.п.с. }V = V_{\lambda_1}\oplus V_{\lambda_2} = span(V_1, V_2, V_3)\\
		T_{e\rightarrow v} = \begin{pmatrix}
			2 & 1 & 3\\
			1 & 0 & 5\\
			0 & -1 & 6
		\end{pmatrix}\\
		T^{-1}AT = \begin{pmatrix}
			1 & 0 & 0\\
			0 & 1 & 0\\
			0 & 0 & -1
		\end{pmatrix} = \Lambda \ \ \ \boxed{AT = T\Lambda}\\
		\p_1: V\rightarrow V_{\lambda_1}\subset{V} \\
		\p_2: V\rightarrow V_{\lambda_2} \subset{V}\\
		\p_1' \text{ матрица }\p_1 \text{ в базисе } v = \begin{pmatrix}
			1 & 0 & 0\\
			0 & 1 & 0\\
			0 & 0 & 0
		\end{pmatrix}\\
		\p_1, \p_2 $-- матрицы проекторов в базисе $e$(канонич.)\\
		$\p_1 v_i = \left[\begin{array}{l}
			v_i, i = 1, 2\\
			\0, i = 3
		\end{array}\right.\\
		1^\circ \ 2^\circ \ 3^\circ\\
		\p_1' + \p_2' = E\\
		\p_1'\p_2' = \0 \ldots\\
		\p_2'$ матрица $\p_2$ в базисе $v = \begin{pmatrix}
			0 & 0 & 0\\
			0 & 0 & 0\\
			0 & 0 & 1
		\end{pmatrix}$
	\end{examples}
	\begin{examples}
		\belowbaseline[-10pt]{
			$A = \begin{pmatrix}
				7 & -12 & 6\\
				10 & -19 & 10\\
				12 & -24 & 13
			\end{pmatrix}$}\\
		    $\p_i' = T^{-1}\p_i T \ \ \ \ \ i = 1, 2\\
			\p_i = T\p_i'T^{-1} \ \ \ \begin{array}{c}
				\p_1, \p_2 = \0\\
				\p_1^2 = \p_1
			\end{array}\\
			\p_1 = \begin{pmatrix}
				4 & -6 & 3\\
				6 & -9 & 5\\
				6 & -12 & 7
			\end{pmatrix} \ \ \p_2 = \begin{pmatrix}
				-3 & 6 & -3\\
				-5 & 10 & -5\\
				-6 & 12 & -6
			\end{pmatrix} = E - \p_1\\
			$
	\end{examples}
	\begin{defin}
		$(A_k) = ((a^k_{ij}))^\infty_{k=1}$ -- последовательность матриц\\
		$\exists \lim\limits_{k\rightarrow\infty}A_k = A = (a_{ij}) \Leftrightarrow \forall \ i, j \ \exists a_{ij} = \lim\limits_{k\rightarrow\infty} a^k_{ij}$\\
		$$S = \underbracket{\sum\limits_{m=1}^\infty A_m}_{\stackrel{\text{\textlarger[2]{Ряд.}}}{\text{Сумма ряда.}}} \overset{\exists}{=} \lim\limits_{N\rightarrow\infty}\underbracket{\sum\limits_{m=1}^N A_m}_{\stackrel{S_N \text{ \textlarger[2]{частичная}}}{\text{ сумма ряда}}}\\
		$$
		$$
		f(x)\text{ аналитическая в }|x|<R \Leftrightarrow f(x) = \sum_{m=0}^\infty C_m(x)^m \ \ \ C_m = \frac{f^{(m)}(0)}{m!}
		$$
	\end{defin}
	Ряд Тейлора.\\
	$\mathlarger{e^x = \sum\limits_{m=0}^\infty \frac{x^m}{m!} \ R = \infty \ \cos x = \sum\limits_{m=1}^\infty \frac{(-1)^m x^{2m}}{(2m)!} \ \ \ R = \infty}\\
	\mathlarger{
	\ln(1+x) = \sum\limits_{m=1}^\infty \frac{(-1)^{m-1} x^m}{m} \ \ |x| < 1 \ \ \ \text{либо } x = 1}$
	\begin{defin}
		Функция от матрицы.\\
		$A_{n\times n}\\
		f(A) = \sum\limits_{m=0}^\infty C_m A^m$, где $\boxed{\begin{array}{rcl}
			C_m & = & \frac{f^{(m)}(0)}{m!}\\
			f(x) & = & \sum\limits_{m=0}^\infty C_m x^m
		\end{array}}$
	\end{defin}
	$e^A = \sum\limits_{m=0}^\infty \frac{A^m}{m!}\\
	\cos A = \sum\limits_{m=0}^\infty \frac{(-1)^m}{(2m)!}A^{2m}$
	\begin{theorem}
		$f$ аналитическая в $|x| < R\\
		A_{n\times n} \ \ $ все с.ч. $|\lambda| < R$\\
		\textbf{$A$ диагонализируемая} То есть:\\
		$\exists \underset{\text{невырожд.}}{T}: \Lambda = T^{-1}A T\\
			\exists \p_\lambda: A = \sum\limits_{\lambda}\lambda \p_\lambda$\\
		$\Downarrow$
		\begin{mylist}
			\item 
			$\underset{f(A)}{\exists} = T\begin{pmatrix}
				f(\lambda_1) & \ldots & 0\\
				\vdots & \ddots & \vdots\\
				0 & \ldots & f(\lambda_n)
			\end{pmatrix} T^{-1}$
			\item 
			$\underset{f(A)}{\exists} = \sum\limits_{\lambda\text{с.ч.}}f(\lambda)\p_\lambda$
		\end{mylist}
	\end{theorem}	
	\begin{proof}
		\ \\
		\begin{mylist}
			\item 
			\belowbaseline[-10pt]{
			\begin{minipage}{0.3\textwidth}
				$f(A) = \sum\limits_{m=0}^\infty C_m A^m\\
				A^m = (T\Lambda T^{-1})^m = $
			\end{minipage}
			\begin{minipage}{0.4\textwidth}
				$\boxed{\begin{array}{l}f(x) = \sum\limits_{m=0}^\infty C_m x^m\\
				|x|<R
				\end{array}}$
			\end{minipage}}\\\\
			$= T\Lambda \underbrace{T^{-1} T}_{E} \Lambda T^{-1}\ldots T \Lambda T^-1 = \\
			= T\Lambda^m T^{-1} = T\begin{pmatrix}
			\lambda_1^m & 0\\
			0 && \lambda_n^m
			\end{pmatrix} T^{-1}\\
			f(A) = \sum\limits_{m=0}^\infty C_m T\Lambda^m T^{-1} = T(\sum\limits_{m=0}^\infty C_m\Lambda^m) T^{-1} = \\
			= T\begin{pmatrix}
				\sum\limits_{m=0}^\infty C_m \lambda^m_1 & \ldots & 0\\
				\vdots & \ddots & \vdots\\
				0 & \ldots & \sum\limits_{m=0}^\infty C_m\lambda^m_n
			\end{pmatrix} T^{-1} =
			 T\begin{pmatrix}
				 f(\lambda_1) & 0\\
				 0 & f(\lambda_n)
			 \end{pmatrix} T^{-1}\\
			 |\lambda_i| < R
			$
		\item 
		$A^m = (\sum\limits_{\lambda\text{с.ч.}}\lambda\p_\lambda)^m \underset{\stackrel{\mathlarger{\lambda\neq \mu}}{\p_\lambda \p_\mu - \0}}{=} \sum\limits_{\lambda}\lambda^m\p^m_\lambda = \sum\limits_{\lambda}\lambda^m \p_\lambda\\
		f(A) = \sum\limits_{m=0}^\infty C_m(\sum\limits_\lambda \lambda^m \p_\lambda
		) = \sum\limits_\lambda (\underset{|\lambda| < R}{\sum\limits_{m=0}^{\infty}C_m\lambda^m = f(\lambda)})\p_\lambda = \sum\limits_\lambda f(\lambda) \p_\lambda$
	\end{mylist}
	\end{proof}
	\begin{remark}
		$A$ диагон. $\Leftrightarrow A = T\Lambda T^{-1}\\
		\Leftrightarrow A = \sum\limits_{\lambda\text{с.ч.}}\lambda\p_\lambda\\
		f(x) = \sum\limits_{m=0}^\infty c_m x^m\\
		f(A) = T\begin{pmatrix}
			f(\lambda_1) & 0\\
			0 & f(\lambda_n)
		\end{pmatrix} T^{-1}\\
		f(A) = \sum\limits_{\lambda\text{с.ч.}}f(\lambda)\p_\lambda\\
		t\in \R\\
		f(At) = \sum\limits_{m=0}^\infty C_m A^m t^m\\
		t^m A^m = t^m T\Lambda^m T^{-1} = T\begin{pmatrix}
			(\lambda_1 t)^m & 0\\
			0 & f(\lambda_n t)
		\end{pmatrix} T^{-1}\\
		\boxed{
			f(At) = T\begin{pmatrix}
				f(\lambda_1 t) & 0\\
				0 & f(\lambda_n t)
			\end{pmatrix}T^{-1}
		}\\\\
		t^m A^m = \sum\limits_{\lambda\text{с.ч.}}t^m\lambda^m \p_\lambda\\
		\boxed{f(At) = \sum\limits_{\lambda\text{с.ч.}}f(t\lambda)\p_\lambda}$
	\end{remark}
	\begin{examples}
		$e^{At}\\
		A = \begin{pmatrix}
			7 & -12 & 6\\
			10 & -19 & 10\\
			12 & -24 & 13
		\end{pmatrix}\\
		\chi(t) = det(A-tE) = (t-1)^2(t+1)\\
		\lambda_1 = 1 \ \ \alpha(\lambda_1) = 2\\
		\lambda_2 = -1 \ \ \alpha(\lambda_2) = 1\\
		V_{\lambda_1}: \begin{pmatrix}
			6 & -12 & 6 & \vline \ 0\\
			10 & -20 & 10 & \vline \ 0\\
			12 & -24 & 12 & \vline \ 0
		\end{pmatrix}
		\\
		V_{\lambda_1} = span \underset{v_1}{\begin{pmatrix}
				2\\1\\0
			\end{pmatrix}} \underset{v_2}{\begin{pmatrix}
				1\\0\\-1
			\end{pmatrix}} \ \ \gamma(\lambda_1) = 2\\
		V_{\lambda_2}: \left(\begin{array}{ccc|c}
			8 & -12 & 6 & 0\\
			10 & -18 & 10 & 0\\
			12 & -24 & 14 & 0
		\end{array}\right)\\
		V_{\lambda_2} = span\underset{v_3}{\begin{pmatrix}
				3\\5\\6
			\end{pmatrix}} \ \ \gamma(\lambda_2) = 1\\
		\forall \ \lambda: \left.\begin{array}{rcl}
			\alpha(\lambda) & = & \gamma(\lambda)\\
			\sum\limits_\lambda \alpha(\lambda) & = & 3	
		\end{array}\right\} \Rightarrow A \text{ диагонализируемая}\\
		T_{e\rightarrow v} = (v_1 v_2 v_3) = \begin{pmatrix}
			2 & 1 & 3\\
			1 & 0 & 5\\
			0 & -1 & 6
		\end{pmatrix}\\
		A = T\begin{pmatrix}
			1 & 0 & 0\\
			0 & 1 & 0\\
			0 & 0 & 1
		\end{pmatrix} T^{-1}\\
		e^{At} = T\begin{pmatrix}
			e^t & 0 & 0\\
			0 & e^t & 0\\
			0 & 0 & e^t
		\end{pmatrix} T^{-1}
		= \begin{pmatrix}
			4e^t - 3e^{-t} & -6e^t + 6e^{-t} & 3e^t-3e^{-t}\\
			5e^t - 5e^{-t} & -9e^t + 10e^{-t} & 5e^t - 5e^{-t}\\
			6e^t - 6e^{-t} & -12 e^t + 12e^{-t} & 7e^t - 6e^{-t}
		\end{pmatrix}\\
		\p_i : V\underset{i = 1, 2}{\rightarrow} V_{\lambda_i} \subset V\\
		\p_1 = T\left(\begin{array}{cc|c}
			1 & 0 &0\\
			0 & 1 &0\\
			\cline{1-2}
			0 &\multicolumn{1}{c}{0} & 0
		\end{array}\right)T^{-1} = \begin{pmatrix}
			4 & -6 & 3\\
			5 & -9 & 5\\
			6 & -12 & 7
		\end{pmatrix} \ \ Im\p_1 = span(v_1, v_2) = V_{\lambda_1}\\
		\p_2 = T\left(\begin{array}{cc|c}
			0 & \multicolumn{1}{c}{0} & 0\\
			0 & \multicolumn{1}{c}{0} & 0\\
			\cline{3-3}
			0 & 0 & 1
		\end{array}\right) T^{-1} = \begin{pmatrix}
			-3 & 6 & -3\\
			-5 & 10 & -5\\
			-6 & 12 & -6
		\end{pmatrix} \ \ Im\p_2 = span(v_3) = V_{\lambda_2}\\
		A = 1\cdot\begin{pmatrix}
			4 & -6 & 3\\
			5 & -9 & 5\\
			6 & -12 & 7
		\end{pmatrix} + (-1)\cdot\begin{pmatrix}
			-3 & 6 & -3\\
			-5 & 10 & -5\\
			-6 & 12 & -6
		\end{pmatrix}\\
		e^{At} = e^t\cdot\begin{pmatrix}
		4 & -6 & 3\\
		5 & -9 & 5\\
		6 & -12 & 7
		\end{pmatrix} + e^{-t}\cdot\begin{pmatrix}
		-3 & 6 & -3\\
		-5 & 10 & -5\\
		-6 & 12 & -6
		\end{pmatrix}\\
		A_{n\times n} \ \ \ \ x = \begin{pmatrix}
			x_1(t)\\\vdots\\ x_n(t)
		\end{pmatrix} \ \ \ \dot x \text{ -- производная}\\
		\dot{x} = \begin{pmatrix}
			\dot x_1 (t)\\
			\dot x_2(t)\\
			\vdots\\
			\dot x_n(t)
		\end{pmatrix}\\
		\boxed{\dot x = Ax} \ \ \ \; \ \ \ x = e^{At}C \ \ \ C = \begin{pmatrix}
			c_1\\
			\vdots\\
			c_n
		\end{pmatrix}\\
		\text{с.л.д.у. с постоянным коэффициентом однородности} \ \ \begin{array}{c}(e^{At})' = Ae^{At}\\e^{A\cdot 0} = E\end{array}\\
		e^{At} = \left(\sum\limits_{\lambda\text{с.ч.}} e^{\lambda t} \p_\lambda\right)' = \underline{\underline{\sum\limits_{\lambda\text{ с.ч.}} \lambda e^{\lambda t} \p_\lambda}}\\\\
		A\cdot e^{At} = \sum\limits_{\mu} \mu \p_\mu \cdot \sum\limits_\lambda e^{\lambda t} \p_\lambda \underset{\mu = \lambda}{=}\underline{\underline{\sum\limits_\lambda \lambda e^{\lambda t}\p_\lambda}}$
	\end{examples}\newpage
	\begin{remark}
		$\exists \ A^{-1} \Leftrightarrow detA \neq 0 \Leftrightarrow$ \belowbaseline[-12pt]{$\begin{array}{l}
				\text{все с.ч. } \lambda\neq 0\\
				\text{(все корни хар. многочлена)}
			\end{array}$
		}
	\end{remark}
	$\pu A$ диагонализируема. Все с.ч. $\lambda\neq 0\\
	A^{-1} = T\Lambda^{-1} T^{-1} = T\begin{pmatrix}
		\frac{1}{\lambda_1} & 0\\
		0 & \frac{1}{lambda_n}
	\end{pmatrix}T^{-1}\\
	\Lambda\Lambda^{-1} = E\\
	AA^{-1} = \underbracket{T\underbracket{\Lambda \underbracket{T^{-1}T}_E\Lambda^{-1}}_E T^{-1}}_E = E\\
	A^{-1} = \sum\limits_{\lambda\text{с.ч.}}\frac{1}{\lambda}\p_\lambda\\
	(AA^{-1} = E \text{ \underline{упр.}})\\
	\sqrt[m]{A} = T\sqrt[m]{\Lambda} T^{-1} = T\begin{pmatrix}
		\sqrt[n]{\lambda_1} & \ldots & 0\\
		\vdots & \ddots & \vdots\\
		0 & \ldots & \sqrt[m]{\lambda_n}
	\end{pmatrix} T^{-1}\\
	\pu \text{все }\lambda_i \geq 0\\
	(m\text{ нечет }\Rightarrow \lambda \text{ любого знака})\\
	(\sqrt[m]{\Lambda})^m = \Lambda\\
	(\sqrt[m]{A})^m = T\underbracket{ \sqrt[m]{\Lambda} \underbracket{T^{-1}T}_E \sqrt[m]{\Lambda} 
	\underbracket{T^{-1} \ldots T}_E \sqrt[m]{\Lambda} }_\Lambda T^{-1} = T\Lambda T^{-1} = A\\
	\boxed{\sqrt[m]{A} = \sum\limits_{\lambda\text{с.ч.}}\sqrt[m]{\lambda} \p_\lambda}\\
	(\text{упр.: }(\sqrt[m]{A})^m = A)
	$
	\begin{examples}
		$A = \begin{pmatrix}
			7 & -12 & 6\\
			10 & -19 & 10\\
			12 & -24 & 13
		\end{pmatrix}\\
		\begin{matrix}\lambda_1 = 1\\\lambda_2 = -1\end{matrix} \ \ A^{-1} = T\begin{pmatrix}
			1 & 0 & 0\\
			0 & 1 & 0\\
			0 & 0 & -1
		\end{pmatrix}T^{-1}\\
		A^{-1} = \mathlarger{\frac{1}{1}}\p_1 + \mathlarger{\frac{1}{(-1)}}\p_2 = \p_1 -\p_2 = A\\
		A^2 = E$
	\end{examples}
	\subsection{Комплексификаци линейного вещ. пространства. Продолжение вещ. линейного оператора.}
	$\A \in End(V) \ \ \ V $ над полем $K$\\
	$\begin{array}{lll}
		\multicolumn{3}{c}{\chi(\lambda) \underset{\lambda \text{ корень}}{= 0}}\\
		\multicolumn{2}{c}{\swarrow} & \searrow \text{III}\\
		\multicolumn{2}{c}{
			\begin{array}{l}
				K = \R/\mathbb{C}\\
				\text{Все корни }\lambda\in K\\
				\text{Т.е. каждый корень с.ч. }\\
				\sum\limits_{\lambda\text{с.ч}} \alpha(\lambda) = n = dim V
			\end{array}
		}
		& \begin{array}{l}
		K = \R\\
		\text{Не все корни вещ.}\\
		\text{т.е. }\exists \lambda \not\in K = \R\\
		\sum\limits_{\text{вещ.} \lambda \text{с.ч.}}\alpha(\lambda) < n = dim V\\
		\A \rightarrow A ?
		\end{array}\\
		\multicolumn{2}{c}{
			\begin{array}{rl}
				\begin{array}{r}
					\text{I}\swarrow\\
					\forall \ \lambda: \gamma(\lambda) = \alpha(\lambda)\\
					\A \text{ -- о.п.с.} \rightarrow A \text{ диагонализир.}
				\end{array} & 
				\begin{array}{l}\\
					\searrow \text{II}\\
					\exists \ \lambda: \gamma (\lambda) < \alpha(\lambda)\\
					\A \text{ не о.п.с.} \\
					\rightarrow A \text{ приводится к Жордановой форме}
				\end{array}
			\end{array}
		}
	\end{array}$
	\begin{defin}
		$V$ -- линейное пространство над $\R\\
		\forall \ x, y \in V \ \ \ v:= x + iy \ \in V_\mathbb{C}\\
		\forall \ v, v' \in V_\mathbb{C}: \ \ \ \ \ \ \ \begin{matrix}
			x = Re \ v\\
			y = Im \ v
		\end{matrix}\\
		\text{Определим}$
		\begin{mylist}
			\item $v = v' \Leftrightarrow \left[\begin{matrix}
				x = x' \in V\\
				y = y'
			\end{matrix}\right.$
			\item 
			$v + v' = \omega = a + bi \in V_\mathbb{C}\Leftrightarrow \left[ \begin{matrix}
				a = x + x' \in V\\
				b = y + y'
			\end{matrix}\right.$
			\item 
			$\forall\ \lambda = \alpha + i\beta \in \mathbb{C}, \ \ \ \alpha, \beta \in \R\\
			a + bi = \omega = \lambda\cdot v \Leftrightarrow(\alpha + i\beta)(x + iy) = \underbrace{\overbrace{\alpha x - \beta y}^{\mathlarger{a \in V}} + i\overbrace{\beta x + \alpha y}^{\mathlarger{b \in V}}}_{\mathlarger{\in V_\mathbb{C}}}
			$
			\item 
			$\forall \ x\in V \leftrightarrow x + i\0 \in V_{\mathbb{C}}\\
			V \subset V_\mathbb{C}\\
			\0 \leftrightarrow \0 + i\0$
		\end{mylist}
	\underline{Упр.:} $V_\mathbb{C}$ -- линейное пространство над $\mathbb{C}$\\
	$\boxed{V_\mathbb{C} \text{ -- комплексификация линейного вещественного пространства } V}$
	\end{defin}
	\begin{stat}
		$e_1\ldots e_n$ базис $V \Rightarrow e_1 \ldots e_n$ базис $V_\mathbb{C}$\\
		Т.е. $dim V = dim V_\mathbb{C} = n\\
		V \subset V_\mathbb{C}\ \ \ \ \ $ структуры над \underline{разными} полями.
	\end{stat}
	\begin{proof}
		$e_1\ldots e_n$ базис $V_\mathbb{C}?$\\
			-- порождающая?\\
			-- линейно независимая?\\
		\begin{mylist}
			\item 
			$\forall \ v\in V_\mathbb{C} \ \ v = x\in V + iy\in V = \sum\limits_{j=1}^n x_j e_j + i\sum\limits_{j=1}^n y_j e_j = \\
			\sum\limits_{j=1}^n \underset{\alpha_j \in \mathbb{C}}{\boxed{x_j + iy_j}}e_j \Rightarrow e_1\ldots e_n $ порождающая.
			\item
			$\sum\limits_{j=1}^n \gamma_j e_j = \0 \ \ \ \begin{matrix}
				\gamma_j \in \mathbb{C}\\
				\gamma_j = \alpha_j + i\beta_j
			\end{matrix}\\
			||\\
			\underbracket{\sum\limits_{j=1}^n\overset{\text{вещ.}}{\alpha_j}e_j}_x +
			i \underbrace{\sum\limits_{j=1}^n \overset{\text{вещ.}}{\beta_j}e_j}_y = \0\\
			\Leftrightarrow \left\{
				\begin{array}{lcc}
					x = \0 = & \sum\limits_{j=1}^n\overset{\text{вещ.}}{\alpha_j}e_j & \Leftrightarrow\\
					y = \0 = & \sum\limits_{j=1}^n \overset{\text{вещ.}}{\beta_j}e_j &\overset{\mathlarger{e_1\ldots e_n\text{ линейно независ.}}}{\Leftrightarrow}
				\end{array}
			\right.
			\left\{\begin{array}{c}
				\forall \ j \ \alpha_j = 0\\
				\forall \ j \ \beta_j = 0
			\end{array}\right. 
			\Leftrightarrow \forall j\ \gamma_j = 0\\\\
			\Rightarrow \underset{\mathlarger{\text{лин. незав.}}}{e_1\ldots e_n}$ в $V_\mathbb{C}$
		\end{mylist}
	\end{proof}
	\begin{defin}
		$z = x + iy \ \ x, y\in V$\\
		\begin{minipage}{0.5\textwidth}
			\textbf{вектор сопряженный к $z$:}\\
			$\vec{z} = x-iy$\\
			$(\vec{\vec z} = z, (\vec{z_1 + z_2}) = \vec z_1 + \vec z_2, \vec{(\lambda z)} = \vec \lambda \vec z)$\\
			\vfill
		\end{minipage}
		\begin{minipage}{0.5\textwidth}
			$z = \begin{pmatrix}
				z_1\\\vdots\\z_n
			\end{pmatrix}\\
			\vec{z} = \begin{pmatrix}
				\vec z_1\\\vec z_2 \\ \vdots \\ \vec z_n
			\end{pmatrix}$
		\end{minipage}\\
	\end{defin}
	\begin{stat}
		$v_1\ldots v_m $ линейно незав. в $V_\mathbb C \Rightarrow \vec v_1\ldots \vec v_m$ линейно независимы в $V_\mathbb C$\\
		Очевидно, $v_1\ldots v_m$ линейно зависимы $\Rightarrow \vec v_1 \ldots \vec v_m$ линейно зависимы.
	\end{stat}
	\begin{proof}\ \\
		$\left.\begin{array}{ll}
			\vec{\sum\limits_{j=1}^m \gamma_j \vec v_j} & = \vec \0 = \0\\
			||\\
			\sum\limits_{j=1}^m \vec \gamma_j \vec{\vec v_j}& = \sum\limits_{j=1}^m \underset{\text{линейно незав.}}{\gamma'_j v_j}
		\end{array}\right| \Leftrightarrow \forall j \ \gamma'_j = 0 = \vec \gamma_j \Leftrightarrow \gamma_j = 0 \\\\
		\Rightarrow$ линейно независим.
	\end{proof}
	$\boxed{rg(v_1\ldots v_m) = rg(\vec v_1 \ldots \vec v_m)}$
	\begin{defin}
		$\A \in End(V)\\
		V_\mathbb{C}\\
		\forall v = x\in V + i\underset{\in V}{y} \in V_\mathbb{C} \ \ \ \A_\mathbb{C} v = \A x \in V + i\underset{\in V }{\A y} \in V_\mathbb{C}\\
		\A_\mathbb{C}: V_\mathbb{C}\rightarrow V_\mathbb{C}\\
		\A_\mathbb{C} \in End(V_\mathbb{C})$\\
		Линейность?
		\begin{mylist}
			\item 
			Аддитивность. $\A_\mathbb{C}(v_1 + v_2) = \A_\mathbb{C} v_1 + \A_\mathbb{C} v_2$\\
			Очевидно, из аддитивности $\A\\
			v_1 + v_2 = (x_1 + x_2) + i(y_1 + y_2)$
			\item Однородность\\
			 $\forall \lambda = \alpha + i\beta \in \mathbb{C} \ \ \ \alpha, \beta \in \R\\
			 \A_\mathbb{C}(\lambda v) = \A_\mathbb{C}((\alpha + i\beta)(x + iy)) = \\
			 = \A_\mathbb{C}((\alpha x - \beta y) + i(\alpha y + \beta x))= \\
			 = \A (\alpha x - \beta y) + i \A(\alpha y + \beta x) = \\
			 = \alpha \A x - \beta \A y + i \alpha \A y + i \beta \A x = \\
			 = (\alpha + i \beta) \A x + i(\alpha + i\beta) \A y = \lambda \A x + i \lambda \A y =\\
			 = \lambda(\A x + i \A y) = \lambda \A_\mathbb{C}v$
		\end{mylist}
		$\A_\mathbb{C}$ -- \textbf{продолжение линейного вещ. оператора $\A$}\\
		с пространства $V$ на его комплексификацию $V_\mathbb{C}$
	\end{defin}
	\textbf{Свойства $\A_\mathbb{C}$:}
	\begin{mylist}
		\item 
		\belowbaseline[-12pt]{
		$\left.
			\begin{array}{c}
				\underset{\text{веществ.}}{e_1\ldots e_n}$ базис $V(V_\mathbb{C})\\
				\A \leftrightarrow A\\
				\A_\mathbb{C} \leftrightarrow A_\mathbb{C}
			\end{array}
		\right\} \Rightarrow A_\mathbb{C} = A$}\\
		Т.е. $\A_\mathbb{C}$ в вещ. базисе имеет вещ. матрицу, совпадающую с матр. $\A$
		\item $\forall z \in V_\mathbb{C} \ \ \vec{\A_\mathbb{C} z} = \A_\mathbb{C}\vec z\\
		z = x + iy \ \ \vec{\A_\mathbb{C} z} = \vec{\underset{\text{вещ.}}{\A x} + i \underset{\text{вещ.}}{\A y}} = \A x - i \A y = \\
		= \A x + i \A (-y) = \A_\mathbb{C}(x-iy) = \A_\mathbb{C} \vec z$
		\item
		\belowbaseline[-12pt]{
		$ 
		\begin{array}{cccc}
			\chi_\A(t) & =&  \chi_{\A_\mathbb{C}}(t) & \pu e_1\ldots e_n \text{базис} V\\
			|| & & || & \A \leftrightarrow A\\
			det(A-tE) & & det(A_\mathbb{C}-tE) & \A_\mathbb{C} \leftrightarrow A_\mathbb{C} = A
		\end{array}$}\\
		Все корни характеристического многочлена $\chi_\A$ являются собственными числами $\A_\mathbb{C}$
		\item 
		$\chi_\A(\lambda) = \chi_{\A_\mathbb{C}}(\lambda) = 0$\\
		Т.к. многочлен с вещ. коэф. $\Rightarrow \vec \lambda$ тоже корень.\\
		$\lambda = \alpha + i\beta \ \ \text{корень } \chi_{\A_\mathbb{C}} \ \ \ \ \chi_{\A_\mathbb{C}}(\vec \lambda) = 0\\
		v \text{ соотв. с.в.}\\
		\Rightarrow \vec v \text { с.в. для }\vec \lambda = \alpha - i\beta\\
		\boxed{\text{для }\A_{\mathbb{C}}: 
			\begin{matrix}dim V_\lambda = dim V_{\vec \lambda} (\text{из утв. 2})\\
			\gamma(\lambda) = \gamma(\vec \lambda)
			\end{matrix}}\\\\
		\A_\mathbb{C} \vec v \underset{\text{св-во 2}}{=} \vec{\A_\mathbb{C} \underset{\stackrel{\uparrow}{\text{с.в. для }\lambda}}{v}} = \vec{\lambda v} = \vec \lambda \vec v \Rightarrow \vec v \text{ с.в. для } \vec \lambda$
	\end{mylist}\ \\
	\textbf{"III": }$\A\in End(V)\\
	V$ над $\R$\\
	$\sum\limits_{\lambda\text{с.ч.}}\alpha(\lambda) < n = dim V$\\
	Т.е. не все корни $\chi_\A$ \textbf{вещ.}\\
	$\rightarrow \text{строим} \A_\mathbb{C}\in End(V_\mathbb{C}) \ \ \ A_\mathbb{C} = A$\\
	Все корни с.ч. $\Rightarrow$ матрица для $A_\mathbb{C}$ будет сведена либо к I, либо к II\\
	\begin{examples}
		\belowbaseline[-12pt]{$A = \begin{pmatrix}
			4 & -5 & 7\\
			1 & -4 & 9\\
			-4 & 0 & 5
			\end{pmatrix}$}\\
		$\chi_A(t) = det(A-tE) = -(t-1)(t^2-4t + 13)\\
		D = -36 < 0\\
		\lambda_1 = 1 \text{ с.ч. }\alpha(\lambda_1) = 1 \ \ \ \ \ \lambda_{2, 3} = 2+\pm i3 \ \alpha(2, 3) = 1\\
		A_\mathbb{C} = A: \lambda_{2, 3} = 2\pm i\\
		\lambda_1 = 1 \ \ V_{\lambda_1} ] span\begin{pmatrix}
			1\\2\\1
		\end{pmatrix}\\
		\lambda_2 = 2+3i \ \ \ \ \ \; 1\leq \gamma(\lambda_2) \leq \alpha(\lambda_2) = 1 \Rightarrow \gamma(\lambda_2) = 1$\\
		Решаем СЛОУ методом Гаусса точно так же, как мы решали для вещ. чисел.\\ Только теперь арифметические операции с комплексными.\\
		$V_{\lambda_2} = span\begin{pmatrix}
			3-3i\\
			5-3i\\
			4
		\end{pmatrix}\\
		\lambda_3 = 2-3i \ \ \; \; V_{\lambda_3} = span\begin{pmatrix}
			3+3i\\
			5+3i\\
			4
		\end{pmatrix} = v_3\\
		\forall \lambda: \ \gamma(\lambda) = \alpha(\lambda) \Rightarrow A_\mathbb{C} = A \text{ диагонализир.}\\
		T_{e\rightarrow v} = \begin{pmatrix}
			1 & 3-3i & 3+3i\\
			2 & 5-3i & 5+3i\\
			1 & 4 & 4
		\end{pmatrix}\\
		T^{-1} A T = \begin{pmatrix}
			1 & 0 & 0\\
			0 & 2+3i & 0\\
			0 & 0 & 2-3i
		\end{pmatrix}T^{-1} = \ldots
		$
	\end{examples}
	\subsection{Минимальный многочлен. Теорема Кэли-Гамильтона}
	\begin{defin}
		Нормализованный (старший коэф. = 1) многочлен $\psi(t)$ называется \\\textbf{аннулятором элемента $v\in V$}, если $\psi(\A) v = \0$
	\end{defin}
	$\psi(t) = t^m + a_{m-1}t^{m-1} + \ldots + a_1 t + a_0\\
	\psi(\A) = \A^t + a_{m-1}\A^{m-1} + \ldots + a_1 \A + a_0 \E \in End(V)\\
	\A^0 = \E\\
	\psi(t) = \prod\limits_{\lambda \text{ корень}}(t-\lambda)^{m(\lambda)}\\
	(\A-\lambda\E)^{m(\lambda)} \cdot (\A - \mu\E)^{m(\mu)} = (\A-\mu\E)^{m(\mu)} \cdot (\A -\lambda\E)^{m(\lambda)}\\
	\A^k\E^r = \E^r\A^k$\\
	Т.е. перестановочны.
	\begin{defin}
		$\psi(t)$ аннулятор элемента $v\in V $ наименьшший возможной степени \\называется \textbf{минимальным аннулятором элемента $v$}
	\end{defin}
	\begin{theorem}[О минимальном аннуляторе элемента]\ \\
		$\A\in End(V)$
		\begin{mylist}
			\item $\forall v \in V\  \exists! $ минимальный аннулятор $v$
			\item $\forall$ аннулятор элемента делится на его минимальный.
		\end{mylist}
	\end{theorem}
	\begin{proof}\ \\
		\begin{mylist}
			\item 
			\begin{mylist}
				\item 
				$\pu v = \0 \ \ \ \; \psi(t) = 1 \ \ \ \ \ \; $ Очевидно, минимальный аннулятор.\\
				$\psi(\A) v = \E v = \0$
				\item 
				$\pu v \neq \0\\
				\underbracket{\underbracket{(\E)v, \A v, \A^2 v, \ldots, \A^{m-1} v,}_{\text{\normalsize{линейно независимая система}}} \A^m v}_{\text{\normalsize{линейно зависимая система}}}$\\\\
				$dim V = n\\
				m\leq n+1\\
				\A^m v = \sum\limits_{k=0}^{m-1} a_k \A^k v\\
				\0 = \A^m v - \sum\limits_{k=0}^{m-1} a_k \A^k v = (\A^m - \sum\limits_{k-0}^{m-1}a_k\A^k)v \leftarrow \text{ Алгоритм}\\
				\psi(t) = t^m - \sum\limits_{k=0}^{m-1} a_k t^k$\\
				Очевидно, по построению это минимальный аннулятор элемента $v$
			\end{mylist}
			\item 
			$\psi_1$ -- аннулятор $v\\
			\psi_1(t) = a(t) \psi(t) + r(t)\\
			deg \ r(t) < deg \ \psi (t)\\
			\0 = \psi_1(\A) v = (a(\A)\psi(\A) + r(\A))v
			= a(\A) \underbracket{\psi(\A)v}_{=\0} + r(\A) v = r(\A) v\Rightarrow \\
			\Rightarrow \left.\begin{matrix}
				r(t) \text{ аннулятор } v\\deg \ r < deg \ \psi
			\end{matrix}\right\} \Rightarrow \text{Противоречие с минимальностью }\psi \Rightarrow\\
			\Rightarrow r(t) \equiv 0 \Rightarrow \psi_1 \vdots \psi$
		\end{mylist}
	\end{proof}
	\begin{defin}
		Нормализованный многочлен $\phi(t)$ \textbf{называется аннулятором } $\A$,\\ если $\phi(\A) = 0\\
		(\Leftrightarrow \forall v \in V \ \ \phi(\A) v = \0)$\\
		Аннулятор $\A$ минимальной степени называется \textbf{минимальный многочленом}
	\end{defin}
	
	\begin{theorem}[о минимальном многочлене]
		$\A \in End(V)$
		\begin{mylist}
			\item $\forall \A \ \exists ! $ минимальный многочлен
			\item $\forall$ аннулятор $\A$ делится на минимальный многочлен
		\end{mylist}
	\end{theorem}
	\begin{proof}\ \\
		$e_1\ldots e_n$ базис $V\\
		\Rightarrow $ по Теореме 1 для $\forall e_j \ \exists! \ \psi_j $ минимальный аннулятор $e_j\\
		\psi_j(\A) e_j = \0\\
		\psi(t) = $ Н.О.К. ($\psi_1\ldots\psi_n$)\\
		$\forall v \in V \ \ \phi(\A) v = \phi(\A) \sum\limits_{i=1}^n v_i e_i = \sum\limits_{i=1}^n v_i \phi(\A) e_i = \\
		= \sum\limits_{i=1}^n v_i \xi_i(\A) \underbracket{\psi_i (\A) e_i}_{=\0} = \0\\
		\phi\vdots \psi_j \Leftrightarrow \phi(t) = \xi_j(t)\psi_j(t)\\\\
		\Rightarrow \phi(\A) = \0 \Rightarrow \phi$ аннулятор $\A$\\
		Давайте покажем, что у $\phi$ степень минимальная.\\
		От противного.\\
		$\exists \phi_1 $  аннулятор $\A \ \ \ \ \pu deg \ \phi_1 < deg \ \phi\\
		\forall e_j : \phi_1(\A) e_j = \0 \Rightarrow \phi_1 $ аннулятор элемента $e_j\overset{\text{по Теореме 1}}{\Rightarrow}\\
		\Rightarrow \underset{\text{аннулятор }e_j}{\phi_1}\vdots \underset{\text{минимальный аннулятор }e_j}{\psi_j} \Rightarrow \phi_1 \vdots \phi \Rightarrow deg \ \phi_1 \geq deg \ \phi$. Противоречие $\Rightarrow\\
		\Rightarrow deg \ \phi$ минимальный $\Rightarrow $ п.2 доказан, т.к. $\forall$ аннулятор $\A\vdots\phi$\\
		\textbf{Единственность?}\\
		$\pu \underset{\begin{matrix}
				\nwarrow \nearrow\\
				\text{нормализов.} \Rightarrow \text{ ст. коэф. 1}
			\end{matrix}}{\phi_1, \phi}$ минимальные аннуляторы одной степени.\\
		$deg(\phi_1 - \phi) < deg(\phi) = deg(\phi_1)\\
		\forall v \in V \ \ \ (\phi_1 - \phi)(\A) v = \underset{=\0}{\phi_1(\A)}v - \underset{=\0}{\phi(\A)}v = \0 \Rightarrow \\
		\Rightarrow \phi_1 - \phi$ аннулятор $\A$ меньшей степени $\Rightarrow$ противоречие \underline{минимальн.} 
	\end{proof}
	\begin{examples}
		$A = \begin{pmatrix}
			0 & 1 & 0\\
			-4 & 4 & 0\\
			-2 & 1 & 2
		\end{pmatrix}\ \ \ \; \ \phi = ? \text{  минимальный многочлен}\\
		e_1  = \begin{pmatrix}
			1\\0\\0
		\end{pmatrix} \ \ \; \ \phi_1 ?\\
		e_1 = \underbracket{\underbracket{\begin{pmatrix}
				1\\0\\0
				\end{pmatrix} \ \ \; \A e_1 = \begin{pmatrix}
					0\\-4\\2
				\end{pmatrix}}_{\text{\normalsize{линейно независ.}}} \ \ \; \ \A^2e_1 = \begin{pmatrix}
					-4\\-16\\-8
				\end{pmatrix}}_{\text{\normalsize{линейно завис.}}}\\
		\A^2 e_1 = -4e_1 + 4\A e_1\\
		\psi_1(t) = t^2 - 4t + 4 = (t-2)^2\\
		e_2 = \underbracket{\underbracket{\begin{pmatrix}
				0\\1\\0
				\end{pmatrix} \ \ \; \A e_2 = \begin{pmatrix}
				1\\4\\1
				\end{pmatrix}}_{\text{\normalsize{линейно независ.}}} \ \ \; \ \A^2e_2 = \begin{pmatrix}
			4\\12\\-4
			\end{pmatrix}}_{\text{\normalsize{линейно завис.}}}\\
		\A^2 e_2 = 4\A e_2 - 4 e_2\\
		\psi_2(t) = t^2 - 4t + 4 = (t-2)^2\\
		\underbracket{e_2 = \underbracket{\begin{pmatrix}
				0\\0\\1
				\end{pmatrix}}_{\text{\normalsize{лин. нез.}}} \ \ \; \A e_3 = \begin{pmatrix}
				0\\0\\2
		\end{pmatrix}}_{\text{\normalsize{линейно завис.}}}\\
		\A e_3 = 2e_3\\
		\psi_3(t) = t-2\\
		\phi(t) = \text{Н.О.К. }((t-2)^2, (t-2)) = (t-2)^2
		$
	\end{examples}
	\begin{theorem}[Кэли-Гамильтона]
		$\A \in End(V)\\
		\underset{\text{характерист. многочлен}}{\chi(t)} = det(\A-t\E) $ -- аннулятор $\A$
	\end{theorem}
	\begin{proof}
		$\chi(\A) = det(\A - \A) = \0$
	\end{proof}
	Я так и не понял это норм доказательство или нет. В любом случае далее идет длинное док-во.
	\begin{proof}
		$\mu$ -- не корень $\chi(t)\\
		det (\A - \mu \E) \neq 0\\
		\Leftrightarrow \exists (\A -\mu \E)^{-1}\\
		e_1\ldots e_n$ базис $v$. $\A \leftrightarrow A\\
		(A-\mu E)^{-1} = \mathlarger{\frac{1}{det(A-\mu E)}}B \leftarrow$ союзная матрица (прис-ная)\\
		$B = (b_{ij}) \ \ \; \ b_{ij} = (-1)^{ij}M_{ij} \leftarrow $ определиитель $(n-1)$-го порядкка $A-\mu E$\\
		Т.е. мн-н степени $n-1$ относительно $\mu$\\
		$B = B_{n-1} \mu^{n-1} + B_{n-2} \mu^{n-2} + \ldots + B_1 \mu + B_0 \\
		\begin{array}{lc}
			det(A-\mu E) \cdot E = & (A-\mu E) (B_{n-1} \mu^{n-1} + \ldots + B_1 \mu + B_0)\\
			|| & \\
			\chi(\mu)\cdot E & \\
			|| & \\
			\sum\limits_{k=0}^n \alpha_k\mu^k \cdot E &
		\end{array}\\
		\begin{array}{rl|l}
			\mu^0: & \alpha_0 E = AB_0 & A^0\\
			\mu^1: & \alpha_1 E = AB_1 - B_0 & A^1\\
			\mu^2: & \alpha_2 E = AB_2 - B_1 & A^2\\
			\ldots & &\\
			\mu^{n-1}: & \alpha_{n-1} E = AB_{n-1} - B_{n-2} & A^{n-1}\\
			\mu^n: & \alpha_n E = -B_{n-1} & A^n
		\end{array}\\
		\chi(\A) = \chi(A) = \sum\limits_{k=0}^n \alpha_k A^k = AB_0 + A^2 B_1 - AB_0 + A^3 B_2 - A^2 B_1 + \ldots + A^n B_{n-1}\\ - A^{n-1} B_{n-2} - A^n B_{n-1} = \0\\
		\chi$ -- аннулятор $\A$
	\end{proof}\newpage
	\begin{theorem}
		$\A\in End(V)$\\
		Множество корнеq характеристического многочлена $\A$ совпадает с \\множеством корней минимального многочлена $\A$ (без учета кратности)
	\end{theorem}
	\begin{proof}
		$\chi(t)$ -- характерист., $\phi(t)$ -- минимальный многочлен.\\
		$"\Leftarrow" \; \; \pu \phi(\lambda) = 0\ \Rightarrow $ т.к. $\chi$ аннулятор $\A$, то по Т-ме 2 $\chi \vdots \phi \Rightarrow \chi(\lambda) = 0\\
		"\Rightarrow" \; \; \pu \chi(\lambda) = 0$
		\begin{mylist}
			\item $\pu \lambda\in K \Rightarrow \lambda$ с.ч. $\A \ \ \ \; \; \exists v\neq \0: \A v = \lambda V \Rightarrow\\
			\Rightarrow (\A-\lambda\E) v = \0 \Rightarrow \psi(t) = (t-\lambda)$ минимальный аннулятор $v$\\
			Т.к. $\phi\vdots\psi \Rightarrow \lambda $ корень $\phi\\
			\phi(\lambda) = 0$
			\item $\lambda \not \in K$ т.е. III случай: $K = \R$\\
			$\exists$ комплексные корни характерист. многочлена.\\
			$V\rightarrow V_\mathbb{C} \ \ \ \; e_1\ldots e_n $ базис $V \rightarrow$ базис $V_\mathbb{C}\\
			\A \rightarrow \A_\mathbb{C} \ \ \ \; \A_\mathbb{C} e_j = \A e_j + i \A \0 = \A e_j\\
			e_j = e_j + i\0\\
			\Rightarrow \forall k \ \A^k_\mathbb{C} e_j = \A^k e_j$ \\
			$\Rightarrow$ Применим алгоритм построения минимального многочлена (Теоремы 1, 2). \\
			Получим, что минимальные многочлены $\A_\mathbb{C}$ и $\A$ совпадают. \\
			$\left.\begin{array}{c}
				\text{Т.е. }\phi \text{ мин. мн-н для } \A_\mathbb{C}\\
				\chi_{\A_\mathbb{C}} = \chi_\A
			\end{array}\right\} \Rightarrow \text{Применим случай а) для }\A_\mathbb{C}\\
			\Rightarrow \lambda \text{ с.ч.} \ \lambda \text{ корень } \phi$
		\end{mylist}
	\end{proof}
	\begin{examples}
		$A = \begin{pmatrix}
			0 & 1 & 0\\
			-4 & 4 & 0\\
			-2 & 1 & 2
		\end{pmatrix}\\
		\chi(t) = \left|\begin{array}{cc|c}
			-t & 1 & 0\\
			-4 & 4-t & 0\\
			\cline{1-2}
			-2 & \multicolumn{1}{c}{1} & 2-t
		\end{array}\right| = (2-t)(t^2 - 4t + 4) = -(t-2)^3$\\
		Корни $\chi: 2$\\
		Корни $\phi: 2$\\
		$\leadsto$ еще один способ найти с.ч. -- \textbf{найти корни многочлена.}
	\end{examples}
	\begin{corollary}\
		\begin{mylist}
			\item $\underset{\text{\normalsize{характер. (аннулятор)}}}{\psi}\vdots\underset{\text{\normalsize{минимальный (аннулятор мин.)}}}{\phi}$
			\item $deg \ \phi = n = dim V \Rightarrow (-1)^n \chi = \phi\\
			\boxed{\begin{array}{rcl}
					\chi(t) & = & \prod\limits_\lambda (t-\lambda)^{\alpha(\lambda)}\\
					\phi(t) & = & \prod\limits_\lambda(t-\lambda)^{m(\lambda)}
				\end{array} \ 1\leq m(\lambda) \leq \alpha(\lambda)}$
		\end{mylist}
	\end{corollary}
	\subsection{Операторное разложение единицы. Корневые подпространства.}
	\begin{minipage}{0.5\textwidth}
		$\phi(t) = \prod\limits_\lambda (t-\lambda)^{m(\lambda)}$
	\end{minipage}
	\begin{minipage}{0.5\textwidth}
		$\sum\limits_\lambda m(\lambda) = m\\
		deg \ \phi = m$
	\end{minipage}\\
	$P_{m-1}$ -- линейное пространство многочленов степени не выше $m-1\\
	dimP_{m-1} = m\\
	\phi(t) = \underset{\overset{\mathlarger{\nwarrow \nearrow}}{\text{вз. просты}}}{(t-\lambda)^{m(\lambda)}\phi_\lambda(t)} \ \ \; 
	\begin{matrix}
		\phi_\lambda(t) = \prod\limits_{\mu\neq \lambda} (t-\mu)^{m(\mu)}\\
		\phi_\lambda(\lambda) \neq 0\\
		\phi_\lambda(\mu) = 0\\
		\mu \neq \lambda
	\end{matrix}$
	\begin{defin}
		$I_\lambda = \{p\in P_{m-1} | p\vdots \phi_\lambda\}$\\
		\textbf{Главный идеал}, порожденный многочленом $\phi_\lambda = \\
		= \{f\in P_{m(\lambda)-1}|p = f_\lambda \phi_\lambda  \}\\
		I_\lambda$ -- линейное подпространство $P_{m-1}\\
		p_{1, 2}\vdots \phi_\lambda \Rightarrow (p_1 + \alpha p_2)\vdots\phi_\lambda$
	\end{defin}
	\begin{theorem}
		$P_{m-1} = \bigoplus\limits_\lambda I_\lambda$
	\end{theorem}
	\begin{proof}\
		\begin{mylist}
			\item Дизъюнктность.\\
			$\0 = \sum\limits_\lambda \underbrace{f_\lambda\phi_\lambda}_{\in I_\lambda}  = 
			f_\lambda\cdot \phi_\lambda + \underbrace{\sum\limits_{\mu\neq \lambda}f_\mu \underbrace{\phi_\mu}_{\vdots (t-\lambda)^{m(\lambda)}}}_{\vdots (t-\lambda)^{m(\lambda)}}\\
			\Rightarrow f_\lambda \cdot \underset{\text{вз. просты}}{\phi_\lambda \vdots (t-\lambda)^{m(\lambda)}} \Rightarrow \underset{\stackrel{\uparrow}{deg \ f_\lambda = m(\lambda)-1}}{f_\lambda} \vdots (t-\lambda)^{m(\lambda)} \Rightarrow f_\lambda \equiv 0 \\
			\Rightarrow \forall \lambda \ \ \ f_\lambda \equiv 0 \Rightarrow f_\lambda \phi_\lambda \equiv \0 \Rightarrow \text{Дизъюнктны}$
			\item
			\belowbaseline[-12pt]{
			$\begin{array}{lcl}
				dim P_{m-1} & = & m\\
				|| & & \\
				\sum\limits_\lambda dim I_\lambda & = & \sum\limits_\lambda m(\lambda) = m
			\end{array}$}\\\\
			$I_\lambda \subset P_{m-1}\\
			\\
			\Rightarrow P_{m-1} = \bigoplus\limits_\lambda I_\lambda$
		\end{mylist}
	\end{proof}
	\begin{corollary}
		$\forall p \in P_{m-1} \ \exists! \ p = \sum\limits_\lambda p_\lambda\\
		p_\lambda \in I_\lambda\\
		\boxed{1 = \sum\limits_\lambda p_\lambda\text{ -- полиномиальное разложение единицы}}$ 
	\end{corollary}
\end{document}